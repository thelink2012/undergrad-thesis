\xchapter{Programming Interface}{} \label{chap:api}

This appendix documents the interface of JVMTIPROF in detail. It extends upon the documentation already provided by the \jvmtihref{}{JVM Tool Infrastructure (JVMTI)}. Therefore, familiarity with JVMTI's documentation is required to understand the following content.

\medskip
JVMTIPROF must be used in conjunction with a JVMTI environment. A JVMTIPROF environment can be created by using the \hyperref[api:jvmtiProf_Create]{\code{jvmtiProf_Create}} function.

\medskip
Multiple JVMTIPROF environments can be injected into a single JVMTI environment. Each environment retains its own state; for example, each has its logging verbosity, capabilities, and registered events.

\medskip
None of these functions is \jvmtihref{heapCallbacks}{heap callback safe}.


\section{Extension Injection}

%%%%%%%%%%%%%%%%%%%%%%%%%%%%%%%%%%%%%%%%%%%%%%%%%%%%%%%%%%%%%%%%%%%%%%%%%%%%%%%
\begin{apidef}{Create}
\label{api:jvmtiProf_Create}
\begin{lstlisting}[language=C]
jvmtiProfError jvmtiProf_Create(
            JavaVM* vm,
            jvmtiEnv* jvmti_env,
            jvmtiProfEnv** jvmtiprof_env_ptr,
            jvmtiProfVersion version);
\end{lstlisting}

\begin{apidesc}
Creates a JVMTIPROF environment and injects it into the given JVMTI environment.

\medskip
This function is on the global scope, i.e., not in the environment function table.

\medskip
Functionalities of \code{jvmti_env} must not be used before invoking this function.
\end{apidesc}

\begin{apiphase}
\apiphaseonload
\end{apiphase}

\begin{apicap}
\apicaprequired
\end{apicap}

\begin{apiparam}
\apiparamdef{vm}{The virtual machine instance on which the \code{jvmti_env} is installed in.}
\apiparamdef{jvmti_env}{The JVMTI environment to inject JVMTIPROF into.}
\apiparamdef{jvmtiprof_env_ptr}{Pointer through which the injected JVMTIPROF environment pointer is returned.}
\apiparamdef{version}{The version of the JVMTIPROF interface to be created.}
\end{apiparam}

\begin{apierror}
\apierrordef{JVMTIPROF_ERROR_WRONG_PHASE}{The virtual machine is not in the OnLoad phase.}
\apierrordef{JVMTIPROF_ERROR_UNSUPPORTED}{Either the provided JVMTIPROF interface \code{version} is not supported, or the version of the provided \code{jvmti_env} interface is not supported.}
\apierrordef{JVMTIPROF_ERROR_NULL_POINTER}{Either \code{vm}, \code{jvmti_env}, or \code{jvmtiprof_env_ptr} is \code{NULL}.}
\end{apierror}
\end{apidef}
%%%%%%%%%%%%%%%%%%%%%%%%%%%%%%%%%%%%%%%%%%%%%%%%%%%%%%%%%%%%%%%%%%%%%%%%%%%%%%%
%%%%%%%%%%%%%%%%%%%%%%%%%%%%%%%%%%%%%%%%%%%%%%%%%%%%%%%%%%%%%%%%%%%%%%%%%%%%%%%
\begin{apidef}{Get Env}
\label{api:jvmtiProf_GetEnv}
\begin{lstlisting}[language=C]
jvmtiProfError jvmtiProf_GetEnv(
            jvmtiEnv* jvmti_env,
            jvmtiProfEnv** jvmtiprof_env_ptr);
\end{lstlisting}

\begin{apidesc}
Returns the JVMTIPROF environment injected into the given JVMTI environment. 

\medskip
\apiref{Create} must have been previously called on \code{jvmti_env}.

\medskip
This function is on the global scope, i.e., not in the environment function table.
\end{apidesc}

\begin{apiphase}
\apiphaseany
\end{apiphase}

\begin{apicap}
\apicaprequired
\end{apicap}

\begin{apiparam}
\apiparamdef{jvmti_env}{The JMVTI environment to extract JVMTIPROF from.}
\apiparamdef{jvmtiprof_env_ptr}{Pointer through which the injected JMVTIPROF environment pointer is returned.}
\end{apiparam}

\begin{apierror}
\apierrordef{JVMTIPROF_ERROR_NULL_POINTER}{Either \code{jvmti_env} or \code{jvmtiprof_env_ptr} is \code{NULL}.}
\apierrordef{JVMTIPROF_ERROR_INVALID_ENVIRONMENT}{\code{jvmti_env} doesn't have an associated JVMTIPROF environment.}
\end{apierror}
\end{apidef}
%%%%%%%%%%%%%%%%%%%%%%%%%%%%%%%%%%%%%%%%%%%%%%%%%%%%%%%%%%%%%%%%%%%%%%%%%%%%%%%

\section{Method Interception}

%%%%%%%%%%%%%%%%%%%%%%%%%%%%%%%%%%%%%%%%%%%%%%%%%%%%%%%%%%%%%%%%%%%%%%%%%%%%%%%
\begin{apidef}{Set Method Event Flag}
\begin{lstlisting}[language=C]
typedef enum
{
    JVMTIPROF_METHOD_EVENT_ENTRY = 1,
    JVMTIPROF_METHOD_EVENT_EXIT = 2,
} jvmtiProfMethodEventFlag;

jvmtiProfError SetMethodEventFlag(
            jvmtiProfEnv* jvmtiprof_env,
            const char* class_name,
            const char* method_name,
            const char* method_signature,
            jvmtiProfMethodEventFlag flags,
            jboolean enable,
            jint* hook_id_ptr);
\end{lstlisting}

\begin{apidesc}
Sets whether the given method should generate entry/exit events.

\medskip
The \code{jvmtiProfMethodEventFlag} enumeration is a bitmask and can be combined in a single call of this function e.g. \code{JVMTIPROF_METHOD_EVENT_ENTRY | JVMTIPROF_METHOD_EVENT_EXIT}.

\medskip
When \code{JVMTIPROF_METHOD_EVENT_ENTRY} is \code{enable}d, \apieventref{SpecificMethodEntry} events are generated for the given method.

\medskip
When \code{JVMTIPROF_METHOD_EVENT_EXIT} is \code{enable}d, \apieventref{SpecificMethodExit} events are generated for the given method.

\medskip
This function alone does not enable the events. It must be enabled by \apiref{SetEventNotificationMode} and callbacks set in \apiref{SetEventCallbacks}. See appendix~\ref{sec:eventmgr}.
\end{apidesc}

\begin{apiphase}
\apiphaseany
\end{apiphase}

\begin{apicap}
\begin{itemize}
\item \apicapref{can_intercept_methods} is necessary to enable \code{JVMTIPROF_METHOD_EVENT_ENTRY} and \code{JVMTIPROF_METHOD_EVENT_EXIT}.
\end{itemize}
\end{apicap}

\begin{apiparam}
\apiparamdef{class_name}{The name of the class to intercept.}
\apiparamdef{method_name}{The method's name in the given class to intercept.}
\apiparamdef{method_signature}{The signature of the method (i.e. parameter description) to intercept.}
\apiparamdef{flags}{The flags to apply the \code{enable} boolean to.}
\apiparamdef{enable}{Whether to enable (or disable) the events.}
\apiparamdef{hook_id_ptr}{Pointer through which an identifier for the method interception is returned. This identifier can be used in the event to distinguish different intercepted methods.}
\end{apiparam}

\begin{apierror}
\apierrordef{JVMTIPROF_ERROR_NULL_POINTER}{Either \code{class_name}, \code{method_name}, \code{method_signature} or \code{hook_id_ptr} is \code{NULL}.}
\apierrordef{JVMTIPROF_ERROR_ILLEGAL_ARGUMENT}{\code{flags} are not valid. }
\apierrordef{JVMTIPROF_ERROR_MUST_POSSESS_CAPABILITY}{The environment does not possess the capability \apicapref{can_intercept_methods}. Use \apiref{AddCapabilities}.}
\end{apierror}
\end{apidef}
%%%%%%%%%%%%%%%%%%%%%%%%%%%%%%%%%%%%%%%%%%%%%%%%%%%%%%%%%%%%%%%%%%%%%%%%%%%%%%%

\section{Execution Sampling}

%%%%%%%%%%%%%%%%%%%%%%%%%%%%%%%%%%%%%%%%%%%%%%%%%%%%%%%%%%%%%%%%%%%%%%%%%%%%%%%
\begin{apidef}{Set Execution Sampling Interval}
\begin{lstlisting}[language=C]
jvmtiProfError SetExecutionSamplingInterval(
            jvmtiProfEnv* jvmtiprof_env,
            jlong nanos_interval);
\end{lstlisting}

\begin{apidesc}
Sets the CPU-time interval necessary for the \apieventref{SampleExecution} event to happen.

\medskip
Every \code{nano_interval} nanoseconds of CPU-time elapsed causes the \apieventref{SampleExecution} to be called in the thread that expired the timer.

\medskip
This function alone does not enable the event. It must be enabled by \apiref{SetEventNotificationMode} and callbacks set in \apiref{SetEventCallbacks}. See appendix~\ref{sec:eventmgr}.
\end{apidesc}

\begin{apiphase}
\apiphaseany
\end{apiphase}

\begin{apicap}
\begin{itemize}
\item \apicapref{can_generate_sample_execution_events} is necessary for this function to be used.
\end{itemize}
\end{apicap}

\begin{apiparam}
\apiparamdef{nano_interval}{Time interval in nanoseconds necessary for the \apieventref{SampleExecution} event to be invoked.}
\end{apiparam}

\begin{apierror}
\apierrordef{JVMTI_ERROR_ILLEGAL_ARGUMENT}{\code{nanos_interval} is less than \code{0}.}
\apierrordef{JVMTIPROF_ERROR_MUST_POSSESS_CAPABILITY}{The environment does not possess the capability \apicapref{can_generate_sample_execution_events}. Use \apiref{AddCapabilities}.}
\end{apierror}
\end{apidef}
%%%%%%%%%%%%%%%%%%%%%%%%%%%%%%%%%%%%%%%%%%%%%%%%%%%%%%%%%%%%%%%%%%%%%%%%%%%%%%%
%%%%%%%%%%%%%%%%%%%%%%%%%%%%%%%%%%%%%%%%%%%%%%%%%%%%%%%%%%%%%%%%%%%%%%%%%%%%%%%
\begin{apidef}{Get Stack Trace Async}
\begin{lstlisting}[language=C]
typedef struct
{
    jint offset;
    jmethodID method;
} jvmtiProfFrameInfo;

jvmtiProfError GetStackTraceAsync(
            jvmtiProfEnv* jvmtiprof_env,
            jint depth,
            jint max_frame_count,
            jvmtiProfFrameInfo* frame_buffer,
            jint* count_ptr);
\end{lstlisting}

\begin{apidesc}
Get information about the stack of a thread. If \code{max_frame_count} is less than the depth of the stack, the \code{max_frame_count} topmost frames are returned, otherwise the entire stack is returned. The topmost frames, those most recently invoked, are at the beginning of the returned buffer.

\medskip
This function is async-signal-safe and does not require a safepoint, unlike JVMTI's \apijvmtiref{GetStackTrace}.

\medskip
The returned information for each frame contains the \code{method} and bytecode \code{offset} being executed by the frame. If it is impossible to determine \code{offset}, its value is set to a negative value. If set to \code{-1}, a native method is in execution. If set to \code{-2}, the reason for being unable to determine the offset is unknown.

\medskip
The \code{offset} returned in a frame can be used in the bytecode returned by JVMTI's \apijvmtiref{GetBytecodes}. It can also be mapped to a line number by JVMTI's \apijvmtiref{GetLineNumberTable} if and only if JVMTI's \apijvmtiref{GetJLocationFormat} returns \code{JVMTI_JLOCATION_JVMBCI}.

\medskip
See JVMTI's \apijvmtiref{GetStackTrace} for example usage.

\medskip
This function can be used in the \apieventref{SampleExecution} event to obtain the stack trace of the interrupted thread.
\end{apidesc}

\begin{apiphase}
\apiphaselive
\end{apiphase}

\begin{apicap}
\begin{itemize}
\item \apicapref{can_get_stack_trace_asynchronously} is necessary for this function to work.
\end{itemize}
\end{apicap}

\begin{apiparam}
\apiparamdef{depth}{Maximum depth of the call stack trace.}
\apiparamdef{max_frame_count}{Same as \code{depth}.}
\apiparamdef{frame_buffer}{On return, this buffer is filled with stack frame information.}
\apiparamdef{count_ptr}{On return, points to the number of records filled in.}
\end{apiparam}

\begin{apierror}
\apierrordef{JVMTIPROF_ERROR_INTERNAL}{It was not possible to obtain the stack trace.}
\apierrordef{JVMTIPROF_ERROR_ILLEGAL_ARGUMENT}{\code{depth} or \code{max_frame_count} is less than \code{0}.}
\apierrordef{JVMTIPROF_ERROR_ILLEGAL_ARGUMENT}{\code{depth} and \code{max_frame_count} differs.}
\apierrordef{JVMTIPROF_ERROR_NULL_POINTER}{\code{frame_buffer} or \code{count_ptr} is \code{NULL}.}
\end{apierror}
\end{apidef}
%%%%%%%%%%%%%%%%%%%%%%%%%%%%%%%%%%%%%%%%%%%%%%%%%%%%%%%%%%%%%%%%%%%%%%%%%%%%%%%

\section{Event Management} \label{sec:eventmgr}

%%%%%%%%%%%%%%%%%%%%%%%%%%%%%%%%%%%%%%%%%%%%%%%%%%%%%%%%%%%%%%%%%%%%%%%%%%%%%%%
\begin{apidef}{Set Event Notification Mode}
\begin{lstlisting}[language=C]
jvmtiProfError SetEventNotificationMode(
            jvmtiProfEnv* jvmtiprof_env,
            jvmtiEventMode mode,
            jvmtiProfEvent event_type,
            jthread event_thread,
            ...);
\end{lstlisting}

\begin{apidesc}
Controls the generation of events.

\medskip
If \code{JVMTI_ENABLE} is given in \code{mode}, the generation of events of type \code{event_type} are enabled. If \code{JVMTI_DISABLE} is given, the generation of events of \code{event_type} are disabled.

\medskip
If \code{event_thread} is \code{NULL}, the event is enabled or disabled globally; otherwise it is enabled or disabled for a particular thread. An event is generated for a particular thread if it is enabled either globally or at the thread level. No currently implemented event can be controlled at the thread level.

\medskip
See section~\ref{sec:eventmgr} for details on events.

\medskip
Initially no events are enabled at the thread level or global level.

\medskip
Any needed capabilities must be possessed before calling this function.
\end{apidesc}

\begin{apiphase}
\apiphaseonloadlive
\end{apiphase}

\begin{apicap}
Certain capabilities are necessary for some \code{event_type}s:
\begin{itemize}
\item \apicapref{can_generate_sample_execution_events} is necessary for \apieventref{SampleExecution} events.
\item \apicapref{can_intercept_methods} is necessary for \apieventref{SpecificMethodEntry} and \apieventref{SpecificMethodExit} events.
\end{itemize}
\end{apicap}

\begin{apiparam}
\apiparamdef{mode}{Whether to \code{JVMTI_ENABLE} or \code{JVMTI_DISABLE} the event.}
\apiparamdef{event_type}{The event to control.}
\apiparamdef{event_thread}{The thread to control. If \code{NULL}, the event is controlled globally.}
\apiparamdef{...}{For future expansion.}
\end{apiparam}

\begin{apierror}
\apierrordef{JVMTIPROF_ERROR_INVALID_THREAD}{\code{event_thread} is non-\code{NULL} and is not a valid thread}
\apierrordef{JVMTIPROF_ERROR_THREAD_NOT_ALIVE}{\code{event_thread} is non-\code{NULL} and is not live (has not been started or is now dead)}
\apierrordef{JVMTIPROF_ERROR_ILLEGAL_ARGUMENT}{Thread level control was attempted on events that do not allow thread control.}
\apierrordef{JVMTIPROF_ERROR_ILLEGAL_ARGUMENT}{Thread level control was attempted on events that do not allow thread control.}
\apierrordef{JVMTIPROF_ERROR_ILLEGAL_ARGUMENT}{\code{mode} is not a valid \code{jvmtiEventMode}.}
\apierrordef{JVMTIPROF_ERROR_ILLEGAL_ARGUMENT}{\code{event_type} is not a valid \code{jvmtiProfEvent}.}
\apierrordef{JVMTIPROF_ERROR_MUST_POSSESS_CAPABILITY}{The capability necessary to enable the events is not possessed.}
\end{apierror}
\end{apidef}
%%%%%%%%%%%%%%%%%%%%%%%%%%%%%%%%%%%%%%%%%%%%%%%%%%%%%%%%%%%%%%%%%%%%%%%%%%%%%%%
%%%%%%%%%%%%%%%%%%%%%%%%%%%%%%%%%%%%%%%%%%%%%%%%%%%%%%%%%%%%%%%%%%%%%%%%%%%%%%%
\begin{apidef}{Set Event Callbacks}
\begin{lstlisting}[language=C]
jvmtiProfError SetEventCallbacks(
            jvmtiProfEnv* jvmtiprof_env,
            const jvmtiProfEventCallbacks* callbacks,
            jint size_of_callbacks);
\end{lstlisting}

\begin{apidesc}
Sets the functions called for each event. The callbacks are specified by supplying a replacement function table. The function table is copied, changes to the local copy of the table will have no effect. This is an atomic action, all callbacks are set at once. No events are sent before this function is called. When an entry is \code{NULL} or when the event is beyond \code{size_of_callbacks} no event is sent.

\medskip
An event must be enabled and have a callback in order to be sent. The order in which this function and \apiref{SetEventNotificationMode} are called does not affect the result. 

\medskip
See section~\ref{sec:eventmgr} for details on events.
\end{apidesc}

\begin{apiphase}
\apiphaseonloadlive
\end{apiphase}

\begin{apicap}
\apicaprequired
\end{apicap}

\begin{apiparam}
\apiparamdef{callbacks}{The new event callbacks. If \code{NULL}, removes the existing callbacks.}
\apiparamdef{size_of_callbacks}{Must be equal \code{sizeof(jvmtiProfEventCallbacks)}. Used for version compatibility.}
\end{apiparam}

\begin{apierror}
\apierrordef{JVMTI_ERROR_ILLEGAL_ARGUMENT}{\code{size_of_callbacks} is less than \code{0} or not supported.}
\end{apierror}
\end{apidef}
%%%%%%%%%%%%%%%%%%%%%%%%%%%%%%%%%%%%%%%%%%%%%%%%%%%%%%%%%%%%%%%%%%%%%%%%%%%%%%%
%%%%%%%%%%%%%%%%%%%%%%%%%%%%%%%%%%%%%%%%%%%%%%%%%%%%%%%%%%%%%%%%%%%%%%%%%%%%%%%
\iffalse
\begin{apidef}{Generate Events}
\begin{lstlisting}[language=C]
jvmtiProfError GenerateEvents(
            jvmtiProfEnv* jvmtiprof_env,
            jvmtiProfEvent event_type);
\end{lstlisting}

\begin{apidesc}
\end{apidesc}

\begin{apiphase}
\end{apiphase}

\begin{apicap}
\end{apicap}

\begin{apiparam}
\apiparamdef{x}{Y}
\end{apiparam}

\begin{apierror}
\apierrordef{X}{Y}
\end{apierror}
\end{apidef}
\fi
%%%%%%%%%%%%%%%%%%%%%%%%%%%%%%%%%%%%%%%%%%%%%%%%%%%%%%%%%%%%%%%%%%%%%%%%%%%%%%%

\section{Capability}

%%%%%%%%%%%%%%%%%%%%%%%%%%%%%%%%%%%%%%%%%%%%%%%%%%%%%%%%%%%%%%%%%%%%%%%%%%%%%%%
\begin{apidef}{Get Potential Capabilities}
\begin{lstlisting}[language=C]
jvmtiProfError GetPotentialCapabilities(
            jvmtiProfEnv* jvmtiprof_env,
            jvmtiProfCapabilities* capabilities_ptr);
\end{lstlisting}

\begin{apidesc}
Returns the JVMTIPROF features that can potentially be possessed by this environment at this time. 

\medskip
The returned capabilities differ from the complete set of capabilities implemented by JVMTIPROF in two cases: another environment possesses capabilities that can only be possessed by one environment, or the current phase is live, and certain capabilities can only be added during the OnLoad phase. The \apiref{AddCapabilities} function may be used to set any or all of these capabilities. Currently possessed capabilities are included.

\medskip
Typically this function is used in the OnLoad function. Typically, a limited set of capabilities can be added in the live phase. In this case, the set of potentially available capabilities will likely differ from the OnLoad phase set. 

\end{apidesc}

\begin{apiphase}
\apiphaseonloadlive
\end{apiphase}

\begin{apicap}
\apicaprequired
\end{apicap}

\begin{apiparam}
\apiparamdef{capabilities_ptr}{Pointer through which the set of capabilities is returned.}
\end{apiparam}

\begin{apierror}
\apierrordef{JVMTIPROF_ERROR_NULL_POINTER}{\code{capabilities_ptr} is \code{NULL}}
\end{apierror}
\end{apidef}
%%%%%%%%%%%%%%%%%%%%%%%%%%%%%%%%%%%%%%%%%%%%%%%%%%%%%%%%%%%%%%%%%%%%%%%%%%%%%%%


%%%%%%%%%%%%%%%%%%%%%%%%%%%%%%%%%%%%%%%%%%%%%%%%%%%%%%%%%%%%%%%%%%%%%%%%%%%%%%%
\begin{apidef}{Add Capabilities}
\begin{lstlisting}[language=C]
jvmtiProfError AddCapabilities(
            jvmtiProfEnv* jvmtiprof_env,
            jvmtiProfCapabilities* capabilities_ptr);
\end{lstlisting}

\begin{apidesc}
Set new capabilities by adding the capabilities whose values are set to \code{1} in \code{*capabilities_ptr}. All previous capabilities are retained. Typically this function is used in the OnLoad function. A limited set of capabilities can be added in the live phase. 
\end{apidesc}

\begin{apiphase}
\apiphaseonloadlive
\end{apiphase}

\begin{apicap}
\apicaprequired
\end{apicap}

\begin{apiparam}
\apiparamdef{capabilities_ptr}{Pointer to the capabilities to be added.}
\end{apiparam}

\begin{apierror}
\apierrordef{JVMTIPROF_ERROR_NOT_AVAILABLE}{The desired capabilities are not even potentially available.}
\apierrordef{JVMTIPROF_ERROR_NULL_POINTER}{\code{capabilities_ptr} is \code{NULL}}
\end{apierror}
\end{apidef}
%%%%%%%%%%%%%%%%%%%%%%%%%%%%%%%%%%%%%%%%%%%%%%%%%%%%%%%%%%%%%%%%%%%%%%%%%%%%%%%
%%%%%%%%%%%%%%%%%%%%%%%%%%%%%%%%%%%%%%%%%%%%%%%%%%%%%%%%%%%%%%%%%%%%%%%%%%%%%%%
\begin{apidef}{Relinquish Capabilities}
\begin{lstlisting}[language=C]
jvmtiProfError RelinquishCapabilities(
            jvmtiProfEnv* jvmtiprof_env,
            const jvmtiProfCapabilities* capabilities_ptr);
\end{lstlisting}

\begin{apidesc}
Relinquish the capabilities whose values are set to \code{1} in \code{*capabilities_ptr}. Some implementations may allow only one environment to have a capability. This function releases capabilities so that they may be used by other environments. All other capabilities are retained. The capability will no longer be present in \apiref{GetCapabilities}. Attempting to relinquish a capability that the environment does not possess is not an error. 
\end{apidesc}

\begin{apiphase}
\apiphaseonloadlive
\end{apiphase}

\begin{apicap}
\apicaprequired
\end{apicap}

\begin{apiparam}
\apiparamdef{capabilities_ptr}{Pointer to the capabilities to relinquish.}
\end{apiparam}

\begin{apierror}
\apierrordef{JVMTIPROF_ERROR_NULL_POINTER}{\code{capabilities_ptr} is \code{NULL}}
\end{apierror}
\end{apidef}
%%%%%%%%%%%%%%%%%%%%%%%%%%%%%%%%%%%%%%%%%%%%%%%%%%%%%%%%%%%%%%%%%%%%%%%%%%%%%%%
%%%%%%%%%%%%%%%%%%%%%%%%%%%%%%%%%%%%%%%%%%%%%%%%%%%%%%%%%%%%%%%%%%%%%%%%%%%%%%%
\begin{apidef}{Get Capabilities}
\begin{lstlisting}[language=C]
jvmtiProfError GetCapabilities(
            jvmtiProfEnv* jvmtiprof_env,
            jvmtiProfCapabilities* capabilities_ptr);
\end{lstlisting}

\begin{apidesc}
Returns the optional JVMTIPROF features which this environment currently possesses. Each possessed capability is indicated by a \code{1} in the corresponding field of the capabilities structure. An environment does not possess a capability unless it has been successfully added with \apiref{AddCapabilities}. An environment only loses possession of a capability if it has been relinquished with \apiref{RelinquishCapabilities}. Thus, this function returns the net result of the \apiref{AddCapabilities} and \apiref{RelinquishCapabilities} calls which have been made. 
\end{apidesc}

\begin{apiphase}
\apiphaseany
\end{apiphase}

\begin{apicap}
\apicaprequired
\end{apicap}

\begin{apiparam}
\apiparamdef{capabilities_ptr}{Pointer through which the currently possessed capabilities are returned.}
\end{apiparam}

\begin{apierror}
\apierrordef{JVMTIPROF_ERROR_NULL_POINTER}{\code{capabilities_ptr} is \code{NULL}}
\end{apierror}
\end{apidef}
%%%%%%%%%%%%%%%%%%%%%%%%%%%%%%%%%%%%%%%%%%%%%%%%%%%%%%%%%%%%%%%%%%%%%%%%%%%%%%%

\section{General}

%%%%%%%%%%%%%%%%%%%%%%%%%%%%%%%%%%%%%%%%%%%%%%%%%%%%%%%%%%%%%%%%%%%%%%%%%%%%%%%
\begin{apidef}{Dispose Environment}
\begin{lstlisting}[language=C]
jvmtiProfError DisposeEnvironment(jvmtiProfEnv* jvmtiprof_env);
\end{lstlisting}

\begin{apidesc}
Shutdowns a JVMTIPROF connection created with \apiref{Create}. Disposes of any resources held by the environment. JVMTIPROF hooks on the associated JVMTI environment will be undone.

\medskip
Any capabilities held by this environment are relinquished.

\medskip
Events enabled by this environment will no longer be run. However, event handles currently running will continue to run. Caution must be exercised in the design of event handlers whose environment may be disposed of and thus become invalid during their execution.

\medskip
This environment may not be used after this call.
\end{apidesc}

\begin{apiphase}
\apiphaseany
\end{apiphase}

\begin{apicap}
\apicaprequired
\end{apicap}

\apiparamempty

\apierrorempty
\end{apidef}
%%%%%%%%%%%%%%%%%%%%%%%%%%%%%%%%%%%%%%%%%%%%%%%%%%%%%%%%%%%%%%%%%%%%%%%%%%%%%%%
%%%%%%%%%%%%%%%%%%%%%%%%%%%%%%%%%%%%%%%%%%%%%%%%%%%%%%%%%%%%%%%%%%%%%%%%%%%%%%%
\begin{apidef}{Get Jvmti Env}
\begin{lstlisting}[language=C]
jvmtiProfError GetJvmtiEnv(
            jvmtiProfEnv* jvmtiprof_env,
            jvmtiEnv** jvmti_env_ptr);
\end{lstlisting}

\begin{apidesc}
Obtains the JVMTI environment associated with this JVMTIPROF environment.
\end{apidesc}

\begin{apiphase}
\apiphaseany
\end{apiphase}

\begin{apicap}
\apicaprequired
\end{apicap}

\begin{apiparam}
\apiparamdef{jvmti_env_ptr}{Pointer through which the associated JVMTI environment is returned.}
\end{apiparam}

\begin{apierror}
\apierrordef{JVMTIPROF_ERROR_NULL_POINTER}{\code{jvmti_env_ptr} is \code{NULL}.}
\end{apierror}
\end{apidef}
%%%%%%%%%%%%%%%%%%%%%%%%%%%%%%%%%%%%%%%%%%%%%%%%%%%%%%%%%%%%%%%%%%%%%%%%%%%%%%%
%%%%%%%%%%%%%%%%%%%%%%%%%%%%%%%%%%%%%%%%%%%%%%%%%%%%%%%%%%%%%%%%%%%%%%%%%%%%%%%
\begin{apidef}{Get Version Number}
\begin{lstlisting}[language=C]
jvmtiProfError GetVersionNumber(
            jvmtiProfEnv* jvmtiprof_env,
            jint* version_ptr);
\end{lstlisting}

\begin{apidesc}
Return the JVMTIPROF version.

\medskip
The version identifier follows the same convention as JVMTI, with a major, minor and micro version. These can be accessed through the \code{JVMTI_VERSION_MASK_MAJOR}, \code{JVMTI_VERSION_MASK_MINOR}, and \code{JVMTI_VERSION_MASK_MICRO} bitmasks applied to the returned value. The version identifier does not include an interface type, as such \\ \code{JVMTI_VERSION_MASK_INTERFACE_TYPE} is not used.

\medskip
See JVMTI's \apijvmtiref{GetVersionNumber} for details on the bitmasks.
\end{apidesc}

\begin{apiphase}
\apiphaseany
\end{apiphase}

\begin{apicap}
\apicaprequired
\end{apicap}

\begin{apiparam}
\apiparamdef{version_ptr}{Pointer through which the JVMTIPROF version is returned.}
\end{apiparam}

\begin{apierror}
\apierrordef{JVMTIPROF_ERROR_NULL_POINTER}{\code{version_ptr} is \code{NULL}.}
\end{apierror}
\end{apidef}
%%%%%%%%%%%%%%%%%%%%%%%%%%%%%%%%%%%%%%%%%%%%%%%%%%%%%%%%%%%%%%%%%%%%%%%%%%%%%%%
%%%%%%%%%%%%%%%%%%%%%%%%%%%%%%%%%%%%%%%%%%%%%%%%%%%%%%%%%%%%%%%%%%%%%%%%%%%%%%%
\begin{apidef}{Get Error Name}
\begin{lstlisting}[language=C]
jvmtiProfError GetErrorName(
            jvmtiProfEnv* jvmtiprof_env,
            jvmtiProfError error,
            char** name_ptr);
\end{lstlisting}

\begin{apidesc}
Return the symbolic name for an \hyperref[api:ec]{error code}.

\medskip
For example \code{GetErrorName(env, JVMTIPROF_ERROR_NONE, &err_name)} would return in \code{err_name} the string \code{"JVMTIPROF_ERROR_NONE"}.
\end{apidesc}

\begin{apiphase}
\apiphaseany
\end{apiphase}

\begin{apicap}
\apicaprequired
\end{apicap}

\begin{apiparam}
\apiparamdef{error}{The error code.}
\apiparamdef{name_ptr}{Pointer through which the symbolic name is returned. The pointer must be freed with the associated JVMTI \apijvmtiref{Deallocate} function.}
\end{apiparam}

\begin{apierror}
\apierrordef{JVMTIPROF_ERROR_ILLEGAL_ARGUMENT}{\code{error} is not a valid error code.}
\apierrordef{JVMTIPROF_ERROR_NULL_POINTER}{\code{name_ptr} is \code{NULL}.}
\end{apierror}
\end{apidef}
%%%%%%%%%%%%%%%%%%%%%%%%%%%%%%%%%%%%%%%%%%%%%%%%%%%%%%%%%%%%%%%%%%%%%%%%%%%%%%%
%%%%%%%%%%%%%%%%%%%%%%%%%%%%%%%%%%%%%%%%%%%%%%%%%%%%%%%%%%%%%%%%%%%%%%%%%%%%%%%
\begin{apidef}{Set Verbose Flag}
\begin{lstlisting}[language=C]
jvmtiProfError SetVerboseFlag(
            jvmtiProfEnv* jvmtiprof_env,
            jvmtiProfVerboseFlag flag,
            jboolean value);
\end{lstlisting}

\begin{apidesc}
Control verbose output. This is the output which typically is sent to \code{stderr}.
\end{apidesc}

\begin{apiphase}
\apiphaseany
\end{apiphase}

\begin{apicap}
\apicaprequired
\end{apicap}

\begin{apiparam}
\apiparamdef{flag}{Which verbose flag to set.}
\apiparamdef{value}{New value of the flag.}
\end{apiparam}

\begin{apierror}
\apierrordef{JVMTIPROF_ERROR_ILLEGAL_ARGUMENT}{\code{flag} is not a valid flag. }
\end{apierror}
\end{apidef}
%%%%%%%%%%%%%%%%%%%%%%%%%%%%%%%%%%%%%%%%%%%%%%%%%%%%%%%%%%%%%%%%%%%%%%%%%%%%%%%

\section{Events}

\begin{apidef}{Specific Method Entry}
\begin{lstlisting}[language=C]
void JNICALL
SpecificMethodEntry(
    jvmtiProfEnv* jvmtiprof_env,
    jvmtiEnv* jvmti_env,
    JNIEnv* jni_env,
    jthread thread,
    jint hook_id)
\end{lstlisting}

\begin{apidesc}
This event is generated when Java programming language methods specified in \apiref{SetMethodEventFlag} are called.
\end{apidesc}

\begin{apiphase}
\eventphaselive
\end{apiphase}

\begin{eventtype}
\code{JVMTIPROF_EVENT_SPECIFIC_METHOD_ENTRY}
\end{eventtype}

\begin{apicap}
\apicapref{can_intercept_methods} is necessary for this event to be generated.
\end{apicap}

\begin{apiparam}
\apiparamdef{jni_env}{The JNI environment of the event (current) thread.}
\apiparamdef{thread}{Thread that is entering the method.}
\apiparamdef{hook_id}{Identifier of the method as returned by \apiref{SetMethodEventFlag}.}
\end{apiparam}
\end{apidef}

\begin{apidef}{Specific Method Exit}
\begin{lstlisting}[language=C]
void JNICALL
SpecificMethodExit(
    jvmtiProfEnv* jvmtiprof_env,
    jvmtiEnv* jvmti_env,
    JNIEnv* jni_env,
    jthread thread,
    jint hook_id)
\end{lstlisting}

\begin{apidesc}
This event is generated when Java programming language methods specified in \apiref{SetMethodEventFlag} return to the caller.
\end{apidesc}

\begin{apiphase}
\eventphaselive
\end{apiphase}

\begin{eventtype}
\code{JVMTIPROF_EVENT_SPECIFIC_METHOD_EXIT}
\end{eventtype}

\begin{apicap}
\apicapref{can_intercept_methods} is necessary for this event to be generated.
\end{apicap}

\begin{apiparam}
\apiparamdef{jni_env}{The JNI environment of the event (current) thread.}
\apiparamdef{thread}{Thread that is leaving the method.}
\apiparamdef{hook_id}{Identifier of the method as returned by \apiref{SetMethodEventFlag}.}
\end{apiparam}
\end{apidef}

\begin{apidef}{Sample Execution}
\begin{lstlisting}[language=C]
void JNICALL
SampleExecution(
    jvmtiProfEnv* jvmtiprof_env,
    jvmtiEnv* jvmti_env,
    JNIEnv* jni_env,
    jthread thread)
\end{lstlisting}

\begin{apidesc}
This event is generated when the CPU-time specified in \apiref{SetExecutionSamplingInterval} has elapsed. The thread that caused the timer to expire is interrupted, and this callback is invoked.

\medskip
The operations performed in this callback must be \href{https://man7.org/linux/man-pages/man7/signal-safety.7.html}{async-signal-safe}. It is only limited to primitive operations and a subset of system calls. This implies that the callback must not use standard C library functions or JVMTIPROF functions. It is recommended to publish any necessary information to an async-signal-safe queue and consume it in another thread.

\medskip
As an exception, the \apiref{GetStackTraceAsync} function can be used during this callback to obtain the stack trace of the interrupted thread.
\end{apidesc}

\begin{apiphase}
\eventphaselive
\end{apiphase}

\begin{eventtype}
\code{JVMTIPROF_EVENT_SAMPLE_EXECUTION}
\end{eventtype}

\begin{apicap}
\apicapref{can_generate_sample_execution_events} is necessary for this event to be generated.
\end{apicap}

\begin{apiparam}
\apiparamdef{jni_env}{The JNI environment of the event (current) thread.}
\apiparamdef{thread}{Thread that expired the timer.}
\end{apiparam}
\end{apidef}

\section{Error Codes} \label{api:ec}

The following error codes are equivalent to the ones \jvmtihref{ErrorSection}{presented in the JVMTI documentation}:
\begin{itemize}
\item{\code{JVMTIPROF_ERROR_NONE}}
\item{\code{JVMTIPROF_ERROR_NULL_POINTER}}
\item{\code{JVMTIPROF_ERROR_INVALID_ENVIRONMENT}}
\item{\code{JVMTIPROF_ERROR_WRONG_PHASE}}
\item{\code{JVMTIPROF_ERROR_INTERNAL}}
\item{\code{JVMTIPROF_ERROR_ILLEGAL_ARGUMENT}}
\item{\code{JVMTIPROF_ERROR_UNSUPPORTED_VERSION}}
\item{\code{JVMTIPROF_ERROR_NOT_AVAILABLE}}
\item{\code{JVMTIPROF_ERROR_MUST_POSSESS_CAPABILITY}}
\item{\code{JVMTIPROF_ERROR_UNATTACHED_THREAD}}
\end{itemize}
