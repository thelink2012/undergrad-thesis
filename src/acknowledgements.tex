This thesis is dedicated to the memory of my dear father, Edilson Amorim. He conceded to me the privilege of having a personal computer from a tender age, which greatly enhanced my access to information and related opportunities. I'm also grateful to my caring mother, Jucilene das Mercês, who, despite the social and financial difficulties, was able to raise a decent and literate man.

During my academic years, I have met many individuals. Some have become close friends, and some have parted away, but they all contributed a little bit to my formation. I'm grateful for that. Many thanks to Dr. Vinicius Petrucci for showing me the way of science. The exposure to a research environment was vital to sharpening my critical thinking skills. Thanks to Dr. Maurício Segundo and Dr. Rubisley Lemes for leading UFBA's competitive programming group (GRUPRO). It not only improved my problem-solving skills but also gave me access to opportunities that I could only dream of.

Results presented in this work were obtained using the Chameleon testbed supported by the National Science Foundation of the United States of America.
