\xchapter{Methodology}{}
\label{cap:methodology}

In this chapter, we present the design (Section~\ref{sec:design}) and implementation (Section~\ref{sec:impl}) of JVMTIPROF.

\section{Design} \label{sec:design}

JVMTIPROF follows a similar design to the JVMTI. Native agents create environments, and those environments have capabilities, events, and other functionalities. Once the environment is disposed of, all the associated capabilities are relinquished and events disabled.

\subsection{Startup \& Shutdown}

JVMTIPROF must be used in conjunction with a JVMTI environment. During agent loading, the \apihref{Create}{\code{jvmtiProf_Create}} function can be invoked, which modifies the existing JVMTI environment and returns an accompanying JVMTIPROF environment. JVMTI functionality can continue to be accessed as usual.

During \jvmtihref{shutdown}{agent shutdown}, the JVMTIPROF environment must be disposed of through the \apiref{DisposeEnvironment} function. Unlike JVMTI, the disposal must be done explicitly since the JVM does not know about JVMTIPROF. However, if the JVMTI environment is disposed of explicitly (through its \apijvmtiref{DisposeEnvironment}), the associated JVMTIPROF environment is automatically destroyed.

\subsection{Functionality}

JVMTIPROF provides functionality to intercept individual methods, sample the application execution, and obtain accurate call stack traces.

Similarly to the JVMTI, events can be set through the \apiref{SetEventNotificationMode} and \apiref{SetEventCallbacks} functions. Capabilities necessary for the event to work properly must be added through the \apiref{AddCapabilities} function.

Appendix~\ref{chap:api} presents details on the programming interface. An example agent that samples execution and prints call stack traces is shown in listing~\ref{lst:example_execution_sampling}.

\subsubsection*{Execution Sampling}

To sample the application, an agent must add the \code{can_generate_sample_execution_events} capability and set an event callback for \apieventref{SampleExecution}. The \apiref{SetExecutionSampingInterval} function can adjust the sampling interval.


Call stack traces can be obtained by using \apiref{GetStackTraceAsync}. The \code{can_get_stack_trace_asynchronously} capability must be set. Unlike JVMTI's \apijvmtiref{GetStackTrace}, this function is async-signals-safe and does not require a safepoint.

\lstinputlisting[language=C++,frame=tb,caption=Example agent that uses JVMTIPROF to sample the application and print its call trace. Error handling is omitted for brevity.,label=lst:example_execution_sampling]{src/listing/demo-sample-execution.cpp}.

\subsubsection*{Method Interception}

JVMTI provides the \apijvmtiref{MethodEntry} event to intercept \emph{every} Java method call, significantly degrading application performance. JVMTIPROF introduces a low-overhead alternative capable of intercepting individual methods.

Clients can hook methods of interest by passing their signature to \apiref{SetMethodEventFlag}. Entry and exit events can be controlled independently of each other. The associated event notification, callbacks (i.e. \apieventref{SpecificMethodEntry}, \apieventref{SpecificMethodExit}) and capabilities (i.e. \code{can_intercept_methods}) must also be set.

\section{Implementation} \label{sec:impl}

\subsection{JVMTI Injection}

JVMTIPROF uses events and capabilities from JVMTI to implement some of its functionalities. Method interception, for instance, is performed through JVMTI's \apijvmtiref{ClassFileLoadHook} event. This poses a challenge because an API client would not be able to use that event for other purposes. Therefore JVMTI needs to be modified for its functionality and JVMTIPROF to co-exist.

The changes are achieved by redirecting function pointers in the JVMTI function table to JVMTIPROF-managed functions. The \apijvmtiref{DisposeEnvironment}, \apijvmtiref{SetEventCallbacks}, \apijvmtiref{SetEventNotificationMode}, \apijvmtiref{RetransformClasses}, \apijvmtiref{RedefineClasses}, \apijvmtiref{GetCapabilities}, \apijvmtiref{AddCapabilities} and \apijvmtiref{RelinquishCapabilities} functions are hooked. Patching occurs during \apihref{Create}{\code{jvmtiProf_Create}}.

\subsubsection*{Environment Disposal}

JVMTI's \apijvmtiref{DisposeEnvironment} is hooked such that it also disposes the associated JVMTIPROF environment.

\subsubsection*{Event Management}

The \apijvmtiref{SetEventCallbacks} function is patched such that the client does not replace event callbacks used by JVMTIPROF. Instead, the pointer to the client event handler is stored, and whenever JVMTIPROF's callback is called, the client handler is also invoked.

The \apijvmtiref{SetEventNotificationMode} hook works in a similar manner. It avoids replacing notification modes in use by JVMTIPROF and instead stores them internally. When an event used by both occur, the modes set by the client are inspected to decide whether the callback stored in \apijvmtiref{SetEventCallbacks} should be called.

\subsubsection*{Capability Management}

The capabilities functions are modified to avoid exposing capabilities set by JVMTIPROF to the client. JVMTI's \apijvmtiref{GetCapabilities} should return an empty set of capabilities even though JVMTIPROF has set some of them (e.g. \jvmtihref{jvmtiCapabilities.can_retransform_classes}{\code{can_retransform_classes}}). \apijvmtiref{RelinquishCapabilities} must not relinquish capabilities possessed by JVMTIPROF. The state of relinquished capabilities is maintained internally by JVMTIPROF such that \apijvmtiref{GetCapabilities} can return a view according to the client's expectations. \apijvmtiref{AddCapabilities} is also modified for this purpose.

\subsubsection*{Class Redefinition}

\apijvmtiref{RetransformClasses} and \apijvmtiref{RedefineClasses} may need to be replaced to force allocation of method identifiers after class redefinition and retransformation. This is explained in detail at Section~\ref{sec:impl_callstacktrace}.

\subsection{Method Interception}

JVMTIPROF can notify clients about method calls of interest. This is achieved by instrumenting the method's bytecode to include calls to a JVMTIPROF-managed native function at its prologue and epilogue. The method call event is sent upstream when the native function is called.

Agents set the target methods through \apiref{SetMethodEventFlag}. When the bytecode of the class associated with the method is loaded, it is instrumented to include JVMTIPROF's internal calls. Class loading is intercepted through JVMTI's  \apijvmtiref{ClassFileLoadHook} event. If the class is already loaded, the instrumentation is forced by calling JVMTI's \apijvmtiref{RetransformClasses} on the said class.

\medskip
\begin{lstlisting}[language=Java,frame=tb,escapechar=@,captionpos=b,caption=Example instrumentation applied by method interception. Instrumented code is in green. The \code{sum} method is modified such that JVMTIPROF is notified about entries and exits on it.,label=lst:method_interception_instrumentation]
@\color{patchadd}public class JVMTIPROF \{@
    @\color{patchadd}public static native void onMethodEntry(long hookPtr);@
    @\color{patchadd}public static native void onMethodExit(long hookPtr);@
@\color{patchadd}\}@

public class Example {

    @\color{patchadd}static final long sumHookPtr = /* determined at runtime */;@

    public int sum(int a, int b) {
        @\color{patchadd}JVMTIPROF.onMethodEntry(sumHookPtr);@
        @\color{patchadd}try \{@
            return a + b;
        @\color{patchadd}\} finally \{@
            @\color{patchadd}JVMTIPROF.onMethodExit(sumHookPtr);@
        @\color{patchadd}\}@
    }
}
\end{lstlisting}

An illustration of the instrumentation performed is given in Listing~\ref{lst:method_interception_instrumentation}. JVMTIPROF defines a new class with native methods to communicate back with C++. The method exit notification is enclosed in a \code{try-finally} block, so exceptions do not cause the event to be missed. An internal identifier of the hooked method is passed as an argument to JVMTIPROF, which is sent upstream, enabling clients to identify which method of interest has been invoked.

\subsection{Execution Sampling}

JVMTIPROF provides an event to simplify the setup of sampling profilers. It is implemented by the means of a high-precision CPU timer (\code{CLOCK_PROCESS_CPUTIME_ID}) and a notification signal (\code{SIGPROF}). The timer is set to expire at certain intervals, as set by \apiref{SetExecutionSampingInterval}, and once expired, the application receives a signal at the thread that caused the timer to expire.

The signal handler then sends the event upstream. The event handler must be async-signal-safe, restricting the client to primitive operations and system calls. It must also be careful not to consume much CPU time since it blocks the calling thread, safepoint polling, and receiving other signals. Usually, the handler must push the information of interest (e.g., stack trace) into a queue, which can be consumed, e.g., by a worker thread in a non-async-signal context.

\subsection{Call Stack Trace} \label{sec:impl_callstacktrace}

JVMTI provides the \apijvmtiref{GetStackTrace} function to obtain call stack traces of the application. However, it cannot be used during sampling because of its non-async-signal-safe nature.

JVMTIPROF introduces \apiref{GetStackTraceAsync}, a function capable of taking stack traces in async-signal contexts. This function can take traces at non-safepoints, such as in the sampling event handler.

\subsubsection*{AsyncGetCallTrace}

Oracle's HotSpot JVM provides an unsupported internal API capable of async-signal-safe stack tracing. It is famously known as \code{AsyncGetCallTrace} (AGCT for short). JVMTIPROF uses such API to implement its \apiref{GetStackTraceAsync}. As a downside, this functionality is not supported in other JVM implementations.

AGCT requires all methods identifiers to be pre-allocated because it cannot do so lazily in an async-signal-safe manner. JVMTI's \apijvmtiref{ClassLoad} and \apijvmtiref{ClassPrepare} events are used for this purpose. Additionally, \apijvmtiref{RetransformClasses} and \apijvmtiref{RedefineClasses} are hooked in the JVMTI function table to re-allocate identifiers after class redefinition.

\subsubsection*{Debugging Non-Safepoints}

For AGCT to work, the JIT must generate metadata such as GC maps for every compiled instruction, which is achieved by setting the \code{-XX:+DebugNonSafepoints} command-line option. JVMTIPROF sets this flag by enabling JVMTI's \apijvmtiref{CompileMethodLoad} event. This event has the side-effect of enabling debugging information at non-safepoint locations.

% TODO -XX:+UnlockDiagnosticVMOptions -XX:+DebugNonSafepoints are still necessary if the agent is attached.
% TODO cite https://github.com/openjdk/jdk/blob/2f3e494b80cce8e357ceac9a897c42d7e8f54af5/src/hotspot/share/prims/forte.cpp#L519-L670
% TODO can improve https://github.com/jvm-profiling-tools/async-profiler/blob/55da899511d6e427d87c85f6ef6b08ea6a0c1746/src/profiler.cpp#L343
