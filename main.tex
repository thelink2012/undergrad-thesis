%% Template para dissertacao/tese na classe UFBAthesis
%% versao 1.0
%% (c) 2005 Paulo G. S. Fonseca
%% (c) 2012 Antonio Terceiro
%% (c) 2014 Christina von Flach
%% www.dcc.ufba.br/~flach/ufbathesis

%% Carrega a classe ufbathesis
%% Opcoes: * Idiomas
%%           pt   - portugues (padrao)
%%           en   - ingles
%%         * Tipo do Texto
%%           bsc  - para monografias de graduacao
%%           msc  - para dissertacoes de mestrado (padrao)
%%           qual - exame de qualificacao de mestrado
%%           prop - exame de qualificacao de doutorado
%%           phd  - para teses de doutorado
%%         * Media
%%           scr  - para versao eletronica (PDF) / consulte o guia do usuario
%%         * Estilo
%%           classic - estilo original a la TAOCP (deprecated) - apesar de deprecated, manter esse.
%%           std     - novo estilo a la CUP (padrao)
%%         * Paginacao
%%           oneside - para impressao em face unica
%%           twoside - para impressao em frente e verso (padrao)

% Atenção: Manter 'classic' na declaracao abaixo:
\documentclass[bsc, en, classic, scr, a4paper]{ufbathesis}
% FIXME I added scr manually, should it be kept as print?

%% Preambulo:
\usepackage[utf8]{inputenc}
\usepackage{graphicx}
\usepackage{hyphenat}
\usepackage{booktabs}
\usepackage{pifont}
\usepackage{multirow}
\usepackage{listings} 
\usepackage{colortbl}
\usepackage{xfrac}
\usepackage[printonlyused, withpage]{acronym}

% Custom preamble
\usepackage[T1]{fontenc} % https://tex.stackexchange.com/q/150767
\usepackage{enumitem} % for itemize spacing options
\usepackage{tikz}
\usetikzlibrary{positioning}
\tikzset{block/.style={draw,thick,text width=2cm,minimum height=1cm,align=center},
line/.style={-latex}
}
\lstset{
  basicstyle = \ttfamily,
  % https://tex.stackexchange.com/questions/69346/how-to-deal-with-very-long-lstinline-phrases-like-long-class-names
  % TODO use this only for lstinline!
  literate={A}{A}{1\discretionary{}{}{}}
           {B}{B}{1\discretionary{}{}{}}
           {C}{C}{1\discretionary{}{}{}}
           {D}{D}{1\discretionary{}{}{}}
           {E}{E}{1\discretionary{}{}{}}
           {F}{F}{1\discretionary{}{}{}}
           {G}{G}{1\discretionary{}{}{}}
           {H}{H}{1\discretionary{}{}{}}
           {I}{I}{1\discretionary{}{}{}}
           {J}{J}{1\discretionary{}{}{}}
           {K}{K}{1\discretionary{}{}{}}
           {L}{L}{1\discretionary{}{}{}}
           {M}{M}{1\discretionary{}{}{}}
           {N}{N}{1\discretionary{}{}{}}
           {O}{O}{1\discretionary{}{}{}}
           {P}{P}{1\discretionary{}{}{}}
           {Q}{Q}{1\discretionary{}{}{}}
           {R}{R}{1\discretionary{}{}{}}
           {S}{S}{1\discretionary{}{}{}}
           {T}{T}{1\discretionary{}{}{}}
           {U}{U}{1\discretionary{}{}{}}
           {V}{V}{1\discretionary{}{}{}}
           {W}{W}{1\discretionary{}{}{}}
           {X}{X}{1\discretionary{}{}{}}
           {Y}{Y}{1\discretionary{}{}{}}
           {Z}{Z}{1\discretionary{}{}{}}
           {\_}{\_}{1\discretionary{}{}{}}
}
\definecolor{patchadd}{rgb}{0.0, 0.65, 0.31}
\setcounter{tocdepth}{2}
%%%%%%%%%%%% API Definition Macros %%%%%%%%%%%%

\newcommand*{\code}{\lstinline[breaklines]}
\newcommand{\apiref}[1]{\hyperref[api:#1]{\code{#1}}}
\newcommand{\apihref}[2]{\hyperref[api:#1]{#2}}
\newcommand{\apieventref}[1]{\code{#1}} % TODO
\newcommand{\apicapref}[1]{\code{#1}} % TODO
\newcommand{\apijvmtiref}[1]{\href{https://docs.oracle.com/en/java/javase/11/docs/specs/jvmti.html\##1}{\code{#1}}}
\newcommand{\jvmtihref}[2]{\href{https://docs.oracle.com/en/java/javase/11/docs/specs/jvmti.html\##1}{#2}}
\newcommand{\jnihref}[2]{\href{https://docs.oracle.com/en/java/javase/11/docs/specs/jni/functions.html\##1}{#2}}

\newenvironment{apidef}[1]{
   %\subsubsection*{\detokenize{#1}} \label{api:#1} *
   % TODO fix underline with an additional space after it
   \noindent \textbf{\detokenize{#1}} \label{api:#1}
   % TODO must remove spaces in label about
   \bigskip

   \newenvironment{apidesc}{
      \bigbreak
      \noindent \textbf{DESCRIPTION} \par
   }{
      \par \bigbreak
   }

   \newenvironment{apiphase}{
      \noindent \textbf{PHASES} \par
   }{
      \par \bigbreak
   }
   \newcommand{\apiphaseany}{This function may be called during any phase.}
   \newcommand{\apiphaseonload}{This function may only be called during the OnLoad phase.}
   \newcommand{\apiphaseonloadlive}{This function may only be called during the OnLoad or the live phase.}

   \newenvironment{apicap}{
      \noindent \textbf{CAPABILITIES} \par
   }{
      \par \bigbreak
   }
   \newcommand{\apicaprequired}{Required functionality.}

   \newenvironment{apiparam}{
      \noindent \textbf{PARAMETERS} \par
      \begin{itemize}[itemsep=0pt]
   }{
      \end{itemize}
      \par \bigbreak
   }
   \newcommand{\apiparamempty}{}
   \newcommand{\apiparamdef}[2]{\item[] \code{##1} ##2}

   \newenvironment{apireturn}{
      \noindent \textbf{RETURNS} \par
   }{
      \par \bigbreak
   }
   \newcommand{\apireturnempty}{}

   \newenvironment{apierror}{
      \noindent \textbf{ERROR CODES} \par
      This function returns either a universal error code or one of the following errors:
      \begin{itemize}[itemsep=0pt]
   }{
      \end{itemize}
      \par \bigbreak
   }
   \newcommand{\apierrorempty}{
      \noindent \textbf{ERROR CODES} \par
      This function returns a universal error code.
      \par \bigbreak
   }
   \newcommand{\apierrordef}[2]{\item[] \code{##1}: ##2}
}{
   \par \bigbreak
}



% Universidade
\university{Universidade Federal da Bahia}

% Endereco (cidade)
\address{Salvador}

% Instituto ou Centro Academico
\institute{Instituto de Matem\'{a}tica e Estatística}

% Nome da biblioteca - usado na ficha catalografica
\library{Biblioteca Reitor Mac\^{e}do Costa} %TODO: is this correct?

% Programa de pos-graduacao %TODO: is this correct?
\program{Programa de Graduação em Ciência da Computação}

% Area de titulacao
\majorfield{Ci\^{e}ncia da Computa\c{c}\~{a}o}

% Titulo da dissertacao
\title{JVMTIPROF: An extension to the Java™ Virtual Machine Tool Infrastructure for building profiling agents}

% Data da defesa
% e.g. \date{19 de fevereiro de 2013}
\date{TODO} % TODO
% e.g. \defenseyear{2013}
\defenseyear{TODO} % TODO

% Autor
% e.g. \author{Jose da Silva}
\author{Denilson das Merc\^{e}s Amorim}

% Orientador(a)
% Opcao: [f] - para orientador do sexo feminino
% e.g. \adviser[f]{Profa. Dra. Maria Santos}
\adviser{Vinicius Tavares Petrucci}

% Orientador(a)
% Opcao: [f] - para orientador do sexo feminino
% e.g. \coadviser{Prof. Dr. Pedro Pedreira}
% Comente se nao ha co-orientador
%\coadviser{Nome Completo do CO-ORIENTADOR}

%% Inicio do documento
\begin{document}

\pgcompfrontpage

%% Parte pre-textual
\frontmatter

\presentationpage

%%%%%%%%%%%%%%%%%%%%%%%%%
% Ficha catalografica
%%%%%%%%%%%%%%%%%%%%%%%%%

\authorcitationname{Amorim, Denilson das Merc\^{e}s} % e.g. Terceiro, Antonio Soares de Azevedo
\advisercitationname{Petrucci, Vinicius Tavares} % e.g. Chavez, Christina von Flach Garcia
%\coadvisercitationname{Sobrenome, Nome do CO-ORIENTADOR} % e.g. Mendonca, Manoel Gomes de
\catalogtype{Monografia (Graduação)} % e.g. ``Tese (Doutorado)''

\catalogtopics{TODO} % Listar palavras-chave do trabalho para a FICHA CATALOGRAFICA}, por exemplo, ``1. Complexidade Estrutural. 2. Qualidade de Software 3. Engenharia de Software''
\catalogcdd{XXX.XX} % e.g.  XXX.XX (número nesse formato serah dado pela biblioteca)
\catalogcdu{XXX.XX.XXX} % e.g.  XXX.XX.XXX (idem) 
\catalogingsheet

%%%%%%%%%%%%%%%%%%%%%
% Termo de aprovacaoo
%%%%%%%%%%%%%%%%%%%%%

%TODO
%\approvalsheet{Salvador, 14 de dezembro de 2018}{
%   \comittemember{Prof. Dr. Maurício Pamplona Segundo}{Universidade Federal da Bahia}
%   \comittemember{Prof. Dr. Professor 2}{Universidade Federal da Bahia}
%   \comittemember{Profa. Dra. Professora 3}{Universidade Federal da Bahia}
   % Para mestrado, apenas 3.
   % \comittemember{Prof. Dr. Professor 4}{Universidade HJKL}
   % \comittemember{Profa. Dra. Professora 5}{Universidade QWERTY}
%}

%%%%%%%%%%%%%%%%%%%%%%%%%%%%%%%%%%%%%%%%
% Dedicatoria, Agradecimentos, Epigrafe
%%%%%%%%%%%%%%%%%%%%%%%%%%%%%%%%%%%%%%%%

% Agradecimentos
% Se preferir, crie um arquivo `a parte e o inclua via \include{}
\acknowledgements
%TODO: write acknowledgements
TODO

%%%%%%%%%%%%%%%%%%%%%
% Resumo em Portugues
%%%%%%%%%%%%%%%%%%%%%

\resumo
Construir agentes de perfilamento eficientes para a máquina virtual Java é uma tarefa desafiadora. O trabalho é dificultado pelo fato da maioria dos frameworks de instrumentação de alto nível serem escritos e precisarem ser usados nas fronteiras da linguagem de programação Java. Este trabalho apresenta a JVMTIPROF, uma extensão para a JVM Tool Infrastructure provendo funcionalidades de alto nível para serem utilizadas por agentes de instrumentação e perfilamento. A JVMTIPROF pode ser utilizada para interceptar chamadas a métodos e obter amostras precisas da pilha de chamadas de uma aplicação, tudo em código nativo. Nós avaliamos nossa solução substituindo a JVMTI em um escalonador de frequência guiada por perfil e não encontramos nenhuma penalidade de desempenho significativa.


% Some keywords:
% Instrumentation

% TODO
\begin{keywords}
TODO
\end{keywords}

%%%%%%%%%%%%%%%%%%%
% Resumo em Ingles
%%%%%%%%%%%%%%%%%%%

\abstract
Building efficient profiling agents for the Java Virtual Machine is a challenging task. The job is hardened by most high-level instrumentation frameworks being written in and for use within the boundaries of the Java programming language. This work presents JVMTIPROF, an extension to the JVM Tool Infrastructure providing high-level functionality for instrumentation and profiling agents. JVMTIPROF can be used to intercept method calls and accurately sample the call stack of an application, all in native code. We evaluated our solution by replacing JVMTI in a profile-guided frequency scaling scheme and found no significant performance penalties.

\begin{keywords}
TODO
\end{keywords}

%%%%%%%%%%%%%%%%%%%
% Sumario / Indice
%%%%%%%%%%%%%%%%%%%

% Comente para ocultar
\tableofcontents

% Lista de figuras
% Comente para ocultar
\listoffigures

% Lista de tabelas
% Comente para ocultar
\listoftables


\iffalse
\chapter*{Lista de Siglas}

% Sintaxe da lista de acordo com a documentação do pacote `acronym'
% documentação: http://mirror.unl.edu/ctan/macros/latex/contrib/acronym/acronym.pdf
\begin{acronym}[PGCOMP]
    \acro{PGCOMP}{Programa de Pós-Graduação em Ciência da Computação}
    \acro{CNPq}{Conselho Nacional de Desenvolvimento Centífico e Tecnológico}
\end{acronym}
\fi

%% Parte textual
\mainmatter

%\input{content-sample}
\xchapter{Introduction}{}

Profiling is a commonly used technique in the software industry used to identify application bottlenecks. These bottlenecks may be of different natures, such as of performance, storage or of power-consumption.

Profiling is sometimes accompained by instrumentation. During instrumentation, the application's code is transformed in order to support the profiler. For example, every function may be modified to produce events at their entry and exit points. This way, the profiler can acurrately measure how much time is spent at function.

Profiling suffers from a compromise between accuracy and performance. Too much interference to the application may cause its performance characteristics to change, making the profiling results useless. On the other hand, too little interference mayn't produce enough data for accurate results.

% Describe chapters

\section{Motivation}

Profilers are typically self-contained programs, reporting metrics and events about another application through an user-interface. Some, however, work as a library, providing the functionality necessary to instrument and profile applications. In such case, programmers may execute arbitrary code based on the metrics and events reported to them.

This is useful for a number of reasons. One of them is to build the self-contained profilers themselves. After all, they need to have code reacting to the profiled application in order to report them to the end user-interface. Another reason is to modify the behavior of an application without modifying its source code. This change in behaviour may be related to the profiling metrics or unrelated to them. The former can be exemplified by a process scheduler. The latter, by improving small pieces of an existing application without rebuilding them, which may be impractical in cases where the source code is private, lost, or time demanding to build.

The JVM does provide an API for the purpose of developing program analysis tools. This library is called JVMTI. However, such library provides only bare-bones functionality, leaving it to the programmer to build higher-level functionalities on top of it. For example, in order to instrument a method call, one needs to intercept the loading of bytecode, parse its data structures, modify them, and rebuild it as bytecode.

%The JVM is chosen as the target platform for this work due to its widespread use in server-class applications. An target platform is important in order to focus the efforts of mitigating the performance overhead involved in profiling and instrumentation.
% TODO citations
% CITE: Kafka, Casandra, and others.
% ALSO CITE: According to data from Alibaba’s datacenter, more than 90% of latency-critical cloud services are written and deployed as Java applications [17 - Who Limits the Resource Efficiency of My Datacenter: An Analysis of Alibaba Datacenter Traces]

\section{Related Work}

TODO

% TODO maybe the "JVM does provide an API..." paragraph in the last section goes in the Related Work section.

%% Scheduling
% Compiler-assisted Adaptive Program Scheduling in big.LITTLE Systems
% Hipster: Hybrid Task Manager for Latency-Critical CloudWorkloads


%% TODO profilers and java profilers

%% Profilers with user-customized code
% + An Efficient and Generic Event-based Profiler Frameworkfor Dynamic Languages
% + Portable and accurate samplingprofiling for Java
% + A Portable CPU-ManagementFramework for Java
% + ProfBuilder:  A Package for Rapidly Building Java ExecutionProfilers

%% Path tracing and CCT
% + Exploiting  Hardware  Performance  Counters  with  Flow  and  Context  Sensitive  Profiling 
% A Portable Sampling-Based  Profiler for Java Virtual Machines

%% Async
% https://dl.acm.org/doi/pdf/10.1145/2568088.2576759

% Related:
% Sampling Profiler API for sw/hw counters (remember openmp stuff?)

% Async Profiler
% Coz
% Free-Lunch
% JNIF
% ASM

% PAPER Profiling and Tracing Support for Java Applications

\section{Contribution}

In this work, we present JVMTIPROF, a library extending the JVMTI. We use the same patters, idioms, and types, extending its interface to provide more (and easier) functionalities for profiling and instrumentation agents. This way, developers of these agents can focus their effort on methods instead of in the supporting infrastructure.

% TODO we also contribue by showing how to extend JVMTI



\xchapter{Background}{}

TODO

\section{Instrumentation}

% INSTRUMENTATION MAY BE USED NOT ONLY FOR PROFILING BUT FOR OTHER THINGS (WHAT?)
% CODE COVERAGE

\section{Profiling}

% Profiling Methods https://homepages.cwi.nl/~boncz/msc/2020-ChristianStuart.pdf Section 2.2

% EXPLAIN THAT PROFILERS CAN BE STATISTICAL, INSTRUMENTATION, ETC
% PROFILERS CAN BE OF STORAGE, MEMORY, HEAP, NETWORK, ENERGY, CPU, ETC

% OTHER USES: SCHEDULER, GARBAGE COLLECTOR, JIT COMPILERS

% To sample an application

% signals
% SIGNAL
% ASYNC SIGNAL SAFE
\iffalse
Profilers typically need to sample the call stack of a thread every few milliseconds. A common way to achieve this on Unix-like systems is to have an application-wide timer that notifices its expiration through a signal. The \code{SIGPROF}) signal and \code{ITIMER_PROF} timer is typically used for this purposes. Once the signal is received, the call stack is sampled, by walking on the stack of the running thread.

Since the timer is application-wide, it expires when the total CPU time consumed by threads exceeds the timer value. The signal is then received at the thread that caused the timer to expire. As such, threads that consume more CPU time are more prone to receive the profiling signal than other threads, which is aligned with the profiler's idea of sampling the CPU hungry parts of the application.
\fi


\section{Java Virtual Machine}

% BYTE CODE
% native method
% HotSpot, OpenJ9 etc

% Profiling is used in JITs (may be usefl to write about this?)
% JIT https://homepages.cwi.nl/~boncz/msc/2020-ChristianStuart.pdf Section 2.1.3

\subsection{Java Native Interface}

% Explain COM-like nature (interface pointer)

% contains GetMethodID

\subsection{JVM Tool Interface}

% JVMTI https://homepages.cwi.nl/~boncz/msc/2020-ChristianStuart.pdf Section 2.2.2 etc

\iffalse
Design to the JVMTI. Agents create environments, and those environments have capabilities, events, and other functionalities. Before using any capability that goes beyond the ones provided by a standard JVM, agents must add capabilities to the environment. % TODO example

Function table (\code{jvmtiEnv})

There can be multiple environments

During shutdown it is automatically disposed

% vm phases
% Callback safe

JVMTI provides functionality to intercept all Java methods, but, as expected, that significantly degrades performance. Its documentation suggests
% Cite from docs that it doesn't include full-speed method entry/exit (maybe in introduction/methodology?)

% Explain the docs in few words
% and more
\fi

\subsection{Safepoint Bias}

% Paper: Evaluating the Accuracy of Java Profilers

% SAFEPOINT https://homepages.cwi.nl/~boncz/msc/2020-ChristianStuart.pdf Section 2.2.2 etc
% SAFE POINTS
% https://psy-lob-saw.blogspot.com/2014/03/where-is-my-safepoint.html
% http://psy-lob-saw.blogspot.com/2015/12/safepoints.html
% http://psy-lob-saw.blogspot.com/2016/02/why-most-sampling-java-profilers-are.html
% http://psy-lob-saw.blogspot.com/2016/06/the-pros-and-cons-of-agct.html
% http://blog.ragozin.info/2012/10/safepoints-in-hotspot-jvm.html
% http://jeremymanson.blogspot.com/2013/07/lightweight-asynchronous-sampling.html
% http://jeremymanson.blogspot.com/2007/05/profiling-with-jvmtijvmpi-sigprof-and.html

% AsyncGetCallTrace fails to take some samples https://bugs.openjdk.java.net/browse/JDK-8178287

% The art of JVM Profiling (how async-profiler became to be)
% https://assets.ctfassets.net/oxjq45e8ilak/4mfbX5FJuw0A8M00UK4uKa/ce60f2cab12408e01ce927e90ebb2f7a/Andrey_Pangin__Vadim_Tsesko._The_Art_of_JVM_Profiling.pdf

% JEP draft of AsyncGetCallTrace https://openjdk.org/jeps/8284289

TODO

\section{Elasticsearch}

TODO

\section{Frequency Scaling}

TODO

% big-little
% governors
% DVFS
% see daniel's master thesis 


\iffalse
\section{Elasticsearch}
\fi

\iffalse
\section{Java Virtual Machine}

\subsubsection{Monitors}

\subsubsection{Thread Parking}
\fi

\iffalse
\subsection{Synchronization Primitives}

% Efficient Tracing and Versatile Analysis of Lock Contention in Java Applications on the Virtual Machine Level -- https://dl.acm.org/doi/pdf/10.1145/2851553.2851559
\fi

\iffalse
\subsection{Java Threads}

% http://openjdk.java.net/groups/hotspot/docs/RuntimeOverview.html#Thread%20Management|outline
% https://www.oracle.com/technetwork/java/whitepaper-135217.html

\fi

\iffalse
\section{Process Scheduling}

%single process in the system. In the face of the first problem, time-shared systems were developed to support concurrent users in a single machine \cite{corbato1962experimental}. For the latter, the concept of multi-programming was developed.

%\subsection{Processor core scheduling}

%\subsubsection{Single core processors}

%\subsubsection{Symmetric multi-core processors}

%\subsubsection{Asymmetric multi-core processors}
\fi

\iffalse
\section{Synchronization}

%\subsection{Non-blocking synchronization}

TODO
\fi



% VOCABULARY:
% end-programmer

% EXPLAIN WHY NOT USING JVMTI EXTENSION MECHANISM
% TODO change SetMethodEventFlag

\xchapter{Methodology}{}
\label{cap:methodology}

In this chapter, we present the design (Section~\ref{sec:design}) and implementation (Section~\ref{sec:impl}) of JVMTIPROF.

\section{Design} \label{sec:design}

JVMTIPROF follows a similar design to the JVMTI. Native agents create environments, and those environments have capabilities, events, and other functionalities. Once the environment is disposed of, all the associated capabilities are relinquished and events disabled.

\subsection{Startup \& Shutdown}

JVMTIPROF must be used in conjunction with a JVMTI environment. During \jvmtihref{startup}{agent startup}, when a JVMTI environment is created, a JVMTIPROF environment can be injected into the JVMTI one. This is achieved through the \apihref{Create}{\code{jvmtiProf_Create}} function. This function modifies the JVMTI environment and returns an accompanying function table that can be used to access JVMTIPROF functionality. JVMTI functionality can continue to be accessed through its function table.

During \jvmtihref{shutdown}{agent shutdown}, the JVMTIPROF environment must be disposed of through the \apiref{DisposeEnvironment} function. Unlike JVMTI, the disposal must be done explicitly since the JVM doesn't know about the existence of JVMTIPROF. However, if the JVMTI environment is disposed of explicitly (through its \apijvmtiref{DisposeEnvironment}), the associated JVMTIPROF environment is automatically disposed of. Disposal of the JVMTIPROF environment can be done at any time, not only during agent shutdown.

\subsection{Functionality}

JVMTIPROF provides events that can be used to intercept individual methods and sample the execution of the application.

It also provides functions that can be used to get the accurate call stack trace of a thread.

Similarly to the JVMTI, events can be set through the \apiref{SetEventNotificationMode} and \apiref{SetEventCallbacks} functions. Capabilities necessary for the event to work properly must be added through the \apiref{AddCapabilities} function.

Details on the programming interface can be found in the appendix (Appendix~\ref{chap:api}). An example agent that samples execution and prints call stack traces can be found in listing~\ref{lst:example_execution_sampling}.

\subsubsection*{Execution Sampling}

To sample the application, the agent must enable and set callbacks for the event \apieventref{SampleExecution}, as well as add the \code{can_generate_sample_execution_events} capability. An agent may, optionally, set the sampling interval through the \apiref{SetExecutionSampingInterval} function.


To obtain call stack traces, an agent must possess the necessary capability and invoke \apiref{GetStackTraceAsync}. This function differs from JVMTI's \apijvmtiref{GetStackTrace} in that it doesn't require a safepoint and can be used in async-signals. This avoids the safepoint bias present in most Java profilers.


\lstinputlisting[language=C++,frame=tb,caption=Example agent that uses JVMTIPROF to sample the application and print its call trace. Error handling is omitted for brevity.,label=lst:example_execution_sampling]{src/listing/demo-sample-execution.cpp}.

\subsubsection*{Method Interception}

JVMTI provides functionality to intercept \emph{all} Java methods (i.e. \apijvmtiref{MethodEntry}), but it significantly degrades performance. JVMTIPROF introduces a low overhead alternative capable of intercepting individual methods.

To intercept a Java method, an agent must obtain the method identifier and use \apiref{SetMethodEventFlag} to enable entry and/or exit events on such a method. The method identifier can be obtained through JNI's \jnihref{getmethodid}{\code{GetMethodID}}. Alongside the flag, the associated event notification, callback, and capabilities must be set.

\section{Implementation} \label{sec:impl}

\subsection{JVMTI Injection}

JVMTIPROF uses events and capabilities from JVMTI to implement some of its functionalities. Therefore the JVMTI environment must be modified for JVMTIPROF and JVMTI functionalities to co-exist. For example, method interception is achieved through JVMTI's \apijvmtiref{ClassFileLoadHook} event. As such, an end-programmer wouldn't be able to use the same event for its own purpose, but since we modify JVMTI, an end-programmer can also use events that are in use by JVMTIPROF.

This is achieved by modifying function pointers in the JVMTI function table to point to functions managed by JVMTIPROF. The \apijvmtiref{DisposeEnvironment}, \apijvmtiref{SetEventCallbacks}, \apijvmtiref{SetEventNotificationMode}, \apijvmtiref{RetransformClasses}, \apijvmtiref{RedefineClasses}, \apijvmtiref{GetCapabilities}, \apijvmtiref{AddCapabilities} and \apijvmtiref{RelinquishCapabilities} functions are hooked. This injection process occurs during the \apihref{Create}{\code{jvmtiProf_Create}} function.

\subsubsection*{Environment Disposal}

JVMTI's \apijvmtiref{DisposeEnvironment} is hooked such that it also disposes of the associated JVMTIPROF environment.

\subsubsection*{Event Management}

The \apijvmtiref{SetEventCallbacks} function is hooked such that event callbacks in use by JVMTIPROF don't end up being replaced. Instead, the pointer to these newly set callbacks are stored, and whenever JVMTIPROF's callback for that event is called, the stored callback is also invoked. This way, both JVMTIPROF and the end-programmer can be notified about a JVMTI event.

The \apijvmtiref{SetEventNotificationMode} hook works in a similar manner. It avoids replacing notification modes in use by JVMTIPROF and instead stores them internally. When an event used by both occur, the modes set by the end-programmer are inspected to decide whether the callback stored in \apijvmtiref{SetEventCallbacks} should be called.

\subsubsection*{Capability Management}

The capabilities functions are modified to avoid exposing capabilities set by JVMTIPROF to the end programmer. The JVMTI's \apijvmtiref{GetCapabilities} should return an empty set of capabilities even though JVMTIPROF has set some of them (e.g. \jvmtihref{jvmtiCapabilities.can_retransform_classes}{\code{can_retransform_classes}}). \apijvmtiref{RelinquishCapabilities} must not relinquish capabilities possessed by JVMTIPROF. The state of relinquished capabilities is maintained internally by JVMTIPROF such that \apijvmtiref{GetCapabilities} can return a view that is according to what the end-programmer expects. \apijvmtiref{AddCapabilities} is also modified for this purpose.

% TODO what happens above is kind of a trampoline. Use that word.

\subsubsection*{Class Redefinition}

The \apijvmtiref{RetransformClasses} and \apijvmtiref{RedefineClasses} may need to be hooked to force allocation of method identifiers after class redefinition (or retransformation). This is explained in detail at Section~\ref{sec:impl_callstacktrace}.

\subsection{Method Interception}

JVMTIPROF provides the ability to notify the end-programmer about method calls of interest. This is achieved by instrumenting the bytecode of the method such that its epilogue and prologue include calls to a JVMTIPROF-managed native function. When the JVMTIPROF function is called, the method call event is then sent upstream.

The target methods are set through JVMTIPROF's \apiref{SetMethodEventFlag}, and when the bytecode of the class associated with the target method is being loaded, it is instrumented to include JVMTIPROF's internal calls. Class loading is intercepted through JVMTI's  \apijvmtiref{ClassFileLoadHook} event. If the class is already loaded, the event is forced by calling JVMTI's \apijvmtiref{RetransformClasses} on the class to be instrumented.

\medskip
\begin{lstlisting}[language=Java,frame=tb,escapechar=@,captionpos=b,caption=Example instrumentation applied by method interception. Instrumented code is in green. The \code{sum} method is modified such that JVMTIPROF is notified about entries and exits on it.,label=lst:method_interception_instrumentation]
@\color{patchadd}public class JVMTIPROF \{@
    @\color{patchadd}static native void onMethodEntry(long methodID);@
    @\color{patchadd}static native void onMethodExit(long methodID);@
@\color{patchadd}\}@

public class Example {

    @\color{patchadd}final long sumMethodID = /* determined at runtime */;@

    public int sum(int a, int b) {
        @\color{patchadd}JVMTIPROF.onMethodEntry(sumMethodID);@
        @\color{patchadd}try \{@
            return a + b;
        @\color{patchadd}\} finally \{@
            @\color{patchadd}JVMTIPROF.onMethodExit(sumMethodID);@
        @\color{patchadd}\}@
    }
}
\end{lstlisting}

An illustration of the instrumentation performed is given in Listing~\ref{lst:method_interception_instrumentation}. JVMTIPROF defines a new class with native methods to communicate back with C++. The method exit notification is enclosed in a \code{try...finally} block such that exceptions do not cause the event to be missed. The identifier of the hooked method is passed as an argument to JVMTIPROF since that information is part of the event sent upstream, enabling the end programmer to identify which method has been entered or exited.



% TODO future work: create one JVMTIPROF function dynamically for each hook


\subsection{Execution Sampling}

JVMTIPROF provides an event to simplify the setup of sampling profilers. It is implemented by the means of a high-precision CPU timer (\code{CLOCK_PROCESS_CPUTIME_ID}) and a notification signal (\code{SIGPROF}). The timer is set to expire every interval nanoseconds (as set by \apiref{SetExecutionSampingInterval}), and once expired, the application receives a signal at the thread that caused the timer to expire.

The event handler is then invoked from the signal handler. Since the event is received in an async-signal, its code must be async-signal-safe. That is, it must not perform memory allocations, acquire locks, or perform any other operation that may interfere with the interrupted thread. This restricts the event handler to primitive operations and async-signal-safe system calls. The handler must be careful to not consume much CPU time as well since it is blocking the execution of the thread, safepoint polling, and the receiving of other signals. As such, the handler set by the end-programmer must, in most cases, simply push the sampling information of interest (e.g. stack trace) into a queue, which can be consumed later (e.g. by a sampling thread) in a non-async-signal context.

% TODO future work: per-thread sampling

\subsection{Call Stack Trace} \label{sec:impl_callstacktrace}

JVMTI offers a function to obtain call stack traces of the Java application (\apijvmtiref{GetStackTrace}). However, it cannot be used during execution sampling since it performs non-async-signal-safe operations, such as memory allocations. To mitigate this, some profilers do not use async-signals and instead spawn a thread that takes call traces every few milliseconds.

This tends to produce meaningless profilers for many reasons. First, \apijvmtiref{GetStackTrace} blocks until all of the threads to trace from are in a safepoint. The more threads to trace, the greater the artificial slowdown. Secondly, the traces taken are always at lines of code that are safepoints, implying inaccurate results.

JVMTIPROF introduces \apiref{GetStackTraceAsync}, a function capable of taking stack traces in async-signal contexts. This means traces can be taken at non-safepoints, such as in the execution sampling event handler.

\subsubsection*{AsyncGetCallTrace}

Oracle's HotSpot JVM provides an unsupported internal API capable of async-signal-safe stack tracing. It is famously known as \code{AsyncGetCallTrace} (AGCT for short). JVMTIPROF makes use of such an API to implement its \apiref{GetStackTraceAsync}. As a downside, this functionality isn't supported in other JVM implementations.

AGCT requires all method identifiers to be pre-allocated since it cannot perform memory allocations in an async-signal context. JVMTI's \apijvmtiref{ClassLoad} and \apijvmtiref{ClassPrepare} events are used for this purpose. \apijvmtiref{RetransformClasses} and \apijvmtiref{RedefineClasses} are also hooked in the JVMTI function table to re-allocate identifiers after class redefinition.

\subsubsection*{Debugging Non-Safepoints}

For AGCT to work, the JVM command-line flag \code{-XX:+DebugNonSafepoints} must be present. This option instructs the JVM to produce debug information for lines of code that aren't regarded as safepoints. This debug information allows AGCT to map addresses of JIT compiled code to bytecode locations.

The setting of this flag in JVMTIPROF is achieved by enabling JVMTI's \apijvmtiref{CompiledMethodLoad} event. This event has the side-effect of enabling debugging information on non-safepoints.

% TODO cite https://github.com/openjdk/jdk/blob/2f3e494b80cce8e357ceac9a897c42d7e8f54af5/src/hotspot/share/prims/forte.cpp#L519-L670
% TODO can improve https://github.com/jvm-profiling-tools/async-profiler/blob/55da899511d6e427d87c85f6ef6b08ea6a0c1746/src/profiler.cpp#L343
% TODO -XX:+UnlockDiagnosticVMOptions -XX:+DebugNonSafepoints are still necessary if the agent is attached.

% TODO future work: Windows support




\xchapter{Evaluation}{}
\label{cap:evaluation}

In this chapter, we evaluate JVMTIPROF by using it to instrument Elasticsearch. A profile-guided frequency scaling solution~\cite{hurryupccgrid} is adapted to use JVMTIPROF instead of JVMTI. We then measure the overhead and identify whether JVMTIPROF can produce the same gains as the baseline despite its abstractions.

\section{Experimental Setup}

We conduct experiments on a bare-metal server instance provided by the Chameleon Cloud service. The server consists of an Intel Xeon Gold 6126 (Skylake architecture) with 196GB of DRAM and a 220GB SSD disk. We run Ubuntu 20.04 (Linux kernel 5.4) and Elasticsearch version 6.5.4. For the search index database, we use the Wikipedia dump version \code{enwiki-latest-pages-articles} downloaded in September 2022.

The CPU has 24 physical cores equally divided between two sockets. Intel's Hyperthreading technology is turned off, so we only consider the physical core count. To isolate the network effects in the shared experimental platform, we configure socket 0 to run the server-side applications (Elasticsearch) and socket 1 to execute the load generator (FABAN).

Energy measurement is done through Intel's Running Average Power Limit (RAPL) interface. We measure the energy consumption counter at the beginning and end of the experiment, and the difference between those two values is considered the energy consumed for that particular experiment.

The client, FABAN~\cite{faban}, is responsible for generating query requests with randomized keywords, ranging from 1 to 18 keywords, under a Zipfian distribution.

The load generation process consisted of 20-minute runs of continuous search queries. The FABAN workload generator is used to create four clients, with each client issuing, on average, one request per second. The goal is to analyze the behavior of each solution, considering a single request at a time. An additional configuration of three times the requests-per-seconds is used to explore loads similar to that of production environments.

We perform six runs for each combination of load and solution and report the mean and standard deviation of the 99-percentile latency and energy consumption.

\section{Baseline}

We use the Linux Ondemand power governor and Hurryup~\cite{hurryupccgrid} as baselines for our experiments. Hurryup is a profile-guided frequency scaling solution for Elasticsearch that outperforms Ondemand on energy consumption while attaining acceptable levels of tail latency constraints.

Hurryup works by instrumenting Elasticsearch's hot functions to produce enter and exit events. A runtime scheme uses these events to adapt the individual processor core frequencies. In our experiments, we observe Hurryup saving up to 25\% in energy compared to Linux's Ondemand governor, as seen in Figure~\ref{fig:ondem_vs_hup}.

\begin{figure}[hb]
\centering
\includegraphics[width=1.0\textwidth]{src/figure/ondem_vs_hup.png}
\caption{An experiment comparing Linux's Ondemand governor against Hurryup's frequency scaling scheme. The left plot presents Elasticsearch's service time, and the right plot the energy consumption of the CPU while executing search requests. Results are shown for both a low and high server load.}
\label{fig:ondem_vs_hup}
\end{figure}

\section{Results}

We modify Hurryup to use JVMTIPROF's instrumentation methods and re-execute the experiments as reported in Figure~\ref{fig:ondem_vs_hup_vs_newhup}. We notice that JVMTIPROF can be similar to or slightly worse than JVMTI in the context of Hurryup but can still achieve energy gains compared to Ondemand.

\begin{figure}[ht]
\centering
\includegraphics[width=1.0\textwidth]{src/figure/ondem_vs_hup_vs_newhup.png}
\caption{An experiment comparing Hurryup's performance when implemented with JVMTIPROF versus the baseline. The left plot presents Elasticsearch's service time, and the right plot the energy consumption of the CPU while executing search requests. Results are shown for both a low and high server load.}
\label{fig:ondem_vs_hup_vs_newhup}
\end{figure}

At a lower load, the JVMTI and JVMTIPROF implementations do not present a statistically significant difference in service time and energy consumption. When tripling the load, the JVMTIPROF version can be 6\% worse in service time and 2.78\% worse in energy consumption.

We also compare the overhead of instrumenting Elasticsearch's hot functions, excluding Hurryup-specific logic. Results are presented in Figure~\ref{fig:overhead}. There is no statistically significant difference between instrumenting or not instrumenting said functions with either JVMTI or JVMTIPROF.

\begin{figure}[ht]
\centering
\includegraphics[width=1.0\textwidth]{src/figure/overhead.png}
\caption{An experiment comparing the performance of Elasticsearch when their hot functions are kept as-is or hooked with JVMTI and JVMTIPROF. The left plot presents Elasticsearch's service time, and the right plot the energy consumption of the CPU while executing search requests. Results are shown for both a low and high server load.}
\label{fig:overhead}
\end{figure}

These results also show that the context in which the instrumentation is used may cause performance differences. When instrumenting to run Hurryup at a higher load, JVMTIPROF performed worse than JVMTI, but no difference is seen at lower loads or with scheduling logic removed.

\xchapter{Conclusion}{}
\label{cap:conclusion}

TODO

% FUTURE WORK: more complete Async call stack trace
% FUTURE WORK: ASM interception
% FUTURE WORK: Native method interception
% FUTURE WORK: no need for patching JVMTI (more memory etc but simpler)
% future work: per-thread sampling
% future work: Windows support


%\appendix
\xchapter{Programming Interface}{} \label{chap:api}

This appendix documents the interface of JVMTIPROF in detail. It extends upon the documentation already provided by the \jvmtihref{}{JVM Tool Infrastructure (JVMTI)}. Therefore, familiarity with JVMTI's documentation is required to understand the following content.

\medskip
JVMTIPROF must be used in conjunction with a JVMTI environment. A JVMTIPROF environment can be created by using the \hyperref[api:jvmtiProf_Create]{\code{jvmtiProf_Create}} function.

\medskip
Multiple JVMTIPROF environments can be injected into a single JVMTI environment. Each environment retains its own state; for example, each has its logging verbosity, capabilities, and registered events.

\medskip
None of these functions is \jvmtihref{heapCallbacks}{heap callback safe}.


\section{Extension Injection}

%%%%%%%%%%%%%%%%%%%%%%%%%%%%%%%%%%%%%%%%%%%%%%%%%%%%%%%%%%%%%%%%%%%%%%%%%%%%%%%
\begin{apidef}{Create}
\begin{lstlisting}[language=C]
jvmtiProfError jvmtiProf_Create(
            JavaVM* vm,
            jvmtiEnv* jvmti_env,
            jvmtiProfEnv** jvmtiprof_env_ptr,
            jvmtiProfVersion version);
\end{lstlisting}

\begin{apidesc}
Creates a JVMTIPROF environment and injects it into the given JVMTI environment.

\medskip
This function is on the global scope, i.e., not in the environment function table.

\medskip
Functionalities of \code{jvmti_env} must not be used before invoking this function.
\end{apidesc}

\begin{apiphase}
\apiphaseonload
\end{apiphase}

\begin{apicap}
\apicaprequired
\end{apicap}

\begin{apiparam}
\apiparamdef{vm}{The virtual machine instance on which the \code{jvmti_env} is installed in.}
\apiparamdef{jvmti_env}{The JVMTI environment to inject JVMTIPROF into.}
\apiparamdef{jvmtiprof_env_ptr}{Pointer through which the injected JVMTIPROF environment pointer is returned.}
\apiparamdef{version}{The version of the JVMTIPROF interface to be created.}
\end{apiparam}

\begin{apierror}
\apierrordef{JVMTIPROF_ERROR_WRONG_PHASE}{The virtual machine is not in the OnLoad phase.}
\apierrordef{JVMTIPROF_ERROR_UNSUPPORTED}{Either the provided JVMTIPROF interface \code{version} is not supported, or the version of the provided \code{jvmti_env} interface is not supported.}
\apierrordef{JVMTIPROF_ERROR_NULL_POINTER}{Either \code{vm}, \code{jvmti_env}, or \code{jvmtiprof_env_ptr} is \code{NULL}.}
\end{apierror}
\end{apidef}
%%%%%%%%%%%%%%%%%%%%%%%%%%%%%%%%%%%%%%%%%%%%%%%%%%%%%%%%%%%%%%%%%%%%%%%%%%%%%%%
%%%%%%%%%%%%%%%%%%%%%%%%%%%%%%%%%%%%%%%%%%%%%%%%%%%%%%%%%%%%%%%%%%%%%%%%%%%%%%%
\begin{apidef}{Get Env}
\begin{lstlisting}[language=C]
jvmtiProfError jvmtiProf_GetEnv(
            jvmtiEnv* jvmti_env,
            jvmtiProfEnv** jvmtiprof_env_ptr);
\end{lstlisting}

\begin{apidesc}
Returns the JVMTIPROF environment injected into the given JVMTI environment. 

\medskip
\apiref{Create} must have been previously called on \code{jvmti_env}.

\medskip
This function is on the global scope, i.e., not in the environment function table.
\end{apidesc}

\begin{apiphase}
\apiphaseany
\end{apiphase}

\begin{apicap}
\apicaprequired
\end{apicap}

\begin{apiparam}
\apiparamdef{jvmti_env}{The JMVTI environment to extract JVMTIPROF from.}
\apiparamdef{jvmtiprof_env_ptr}{Pointer through which the injected JMVTIPROF environment pointer is returned.}
\end{apiparam}

\begin{apierror}
\apierrordef{JVMTIPROF_ERROR_NULL_POINTER}{Either \code{jvmti_env} or \code{jvmtiprof_env_ptr} is \code{NULL}.}
\apierrordef{JVMTIPROF_ERROR_INVALID_ENVIRONMENT}{\code{jvmti_env} doesn't have an associated JVMTIPROF environment.}
\end{apierror}
\end{apidef}
%%%%%%%%%%%%%%%%%%%%%%%%%%%%%%%%%%%%%%%%%%%%%%%%%%%%%%%%%%%%%%%%%%%%%%%%%%%%%%%

\section{Method Interception}

%%%%%%%%%%%%%%%%%%%%%%%%%%%%%%%%%%%%%%%%%%%%%%%%%%%%%%%%%%%%%%%%%%%%%%%%%%%%%%%
\begin{apidef}{Set Method Event Flag}
\begin{lstlisting}[language=C]
typedef enum
{
    JVMTIPROF_METHOD_EVENT_ENTRY = 1,
    JVMTIPROF_METHOD_EVENT_EXIT = 2,
} jvmtiProfMethodEventFlag;

jvmtiProfError SetMethodEventFlag(
            jvmtiProfEnv* jvmtiprof_env,
            const char* class_name,
            const char* method_name,
            const char* method_signature,
            jvmtiProfMethodEventFlag flags,
            jboolean enable,
            jint* hook_id_ptr);
\end{lstlisting}

\begin{apidesc}
Sets whether the given method should generate entry/exit events.

\medskip
The \code{jvmtiProfMethodEventFlag} enumeration is a bitmask and can be combined in a single call of this function e.g. \code{JVMTIPROF_METHOD_EVENT_ENTRY | JVMTIPROF_METHOD_EVENT_EXIT}.

\medskip
When \code{JVMTIPROF_METHOD_EVENT_ENTRY} is \code{enable}d, \apieventref{SpecificMethodEntry} events are generated for the given method.

\medskip
When \code{JVMTIPROF_METHOD_EVENT_EXIT} is \code{enable}d, \apieventref{SpecificMethodExit} events are generated for the given method.

\medskip
This function alone does not enable the events. It must be enabled by \apiref{SetEventNotificationMode} and callbacks set in \apiref{SetEventCallbacks}. See appendix~\ref{sec:eventmgr}.
\end{apidesc}

\begin{apiphase}
\apiphaseany
\end{apiphase}

\begin{apicap}
\begin{itemize}
\item \apicapref{can_intercept_methods} is necessary to enable \code{JVMTIPROF_METHOD_EVENT_ENTRY} and \code{JVMTIPROF_METHOD_EVENT_EXIT}.
\end{itemize}
\end{apicap}

\begin{apiparam}
\apiparamdef{class_name}{The name of the class to intercept.}
\apiparamdef{method_name}{The method's name in the given class to intercept.}
\apiparamdef{method_signature}{The signature of the method (i.e. parameter description) to intercept.}
\apiparamdef{flags}{The flags to apply the \code{enable} boolean to.}
\apiparamdef{enable}{Whether to enable (or disable) the events.}
\apiparamdef{hook_id_ptr}{Pointer through which an identifier for the method interception is returned. This identifier can be used in the event to distinguish different intercepted methods.}
\end{apiparam}

\begin{apierror}
\apierrordef{JVMTIPROF_ERROR_NULL_POINTER}{Either \code{class_name}, \code{method_name}, \code{method_signature} or \code{hook_id_ptr} is \code{NULL}.}
\apierrordef{JVMTIPROF_ERROR_ILLEGAL_ARGUMENT}{\code{flags} are not valid. }
\apierrordef{JVMTIPROF_ERROR_MUST_POSSESS_CAPABILITY}{The environment does not possess the capability \apicapref{can_intercept_methods}. Use \apiref{AddCapabilities}.}
\end{apierror}
\end{apidef}
%%%%%%%%%%%%%%%%%%%%%%%%%%%%%%%%%%%%%%%%%%%%%%%%%%%%%%%%%%%%%%%%%%%%%%%%%%%%%%%

\section{Execution Sampling}

%%%%%%%%%%%%%%%%%%%%%%%%%%%%%%%%%%%%%%%%%%%%%%%%%%%%%%%%%%%%%%%%%%%%%%%%%%%%%%%
\begin{apidef}{Set Execution Sampling Interval}
\begin{lstlisting}[language=C]
jvmtiProfError SetExecutionSamplingInterval(
            jvmtiProfEnv* jvmtiprof_env,
            jlong nanos_interval);
\end{lstlisting}

\begin{apidesc}
Sets the CPU-time interval necessary for the \apieventref{SampleExecution} event to happen.

\medskip
Every \code{nano_interval} nanoseconds of CPU-time elapsed causes the \apieventref{SampleExecution} to be called in the thread that expired the timer.

\medskip
This function alone does not enable the event. It must be enabled by \apiref{SetEventNotificationMode} and callbacks set in \apiref{SetEventCallbacks}. See appendix~\ref{sec:eventmgr}.
\end{apidesc}

\begin{apiphase}
\apiphaseany
\end{apiphase}

\begin{apicap}
\begin{itemize}
\item \apicapref{can_generate_sample_execution_events} is necessary for this function to be used.
\end{itemize}
\end{apicap}

\begin{apiparam}
\apiparamdef{nano_interval}{Time interval in nanoseconds necessary for the \apieventref{SampleExecution} event to be invoked.}
\end{apiparam}

\begin{apierror}
\apierrordef{JVMTI_ERROR_ILLEGAL_ARGUMENT}{\code{nanos_interval} is less than \code{0}.}
\apierrordef{JVMTIPROF_ERROR_MUST_POSSESS_CAPABILITY}{The environment does not possess the capability \apicapref{can_generate_sample_execution_events}. Use \apiref{AddCapabilities}.}
\end{apierror}
\end{apidef}
%%%%%%%%%%%%%%%%%%%%%%%%%%%%%%%%%%%%%%%%%%%%%%%%%%%%%%%%%%%%%%%%%%%%%%%%%%%%%%%
%%%%%%%%%%%%%%%%%%%%%%%%%%%%%%%%%%%%%%%%%%%%%%%%%%%%%%%%%%%%%%%%%%%%%%%%%%%%%%%
\begin{apidef}{Get Stack Trace Async}
\begin{lstlisting}[language=C]
typedef struct
{
    jint offset;
    jmethodID method;
} jvmtiProfFrameInfo;

jvmtiProfError GetStackTraceAsync(
            jvmtiProfEnv* jvmtiprof_env,
            jint depth,
            jint max_frame_count,
            jvmtiProfFrameInfo* frame_buffer,
            jint* count_ptr);
\end{lstlisting}

\begin{apidesc}
Get information about the stack of a thread. If \code{max_frame_count} is less than the depth of the stack, the \code{max_frame_count} topmost frames are returned, otherwise the entire stack is returned. The topmost frames, those most recently invoked, are at the beginning of the returned buffer.

\medskip
This function is async-signal-safe and does not require a safepoint, unlike JVMTI's \apijvmtiref{GetStackTrace}.

\medskip
The returned information for each frame contains the \code{method} and bytecode \code{offset} being executed by the frame. If it is impossible to determine \code{offset}, its value is set to a negative value. If set to \code{-1}, a native method is in execution. If set to \code{-2}, the reason for being unable to determine the offset is unknown.

\medskip
The \code{offset} returned in a frame can be used in the bytecode returned by JVMTI's \apijvmtiref{GetBytecodes}. It can also be mapped to a line number by JVMTI's \apijvmtiref{GetLineNumberTable} if and only if JVMTI's \apijvmtiref{GetJLocationFormat} returns \code{JVMTI_JLOCATION_JVMBCI}.

\medskip
See JVMTI's \apijvmtiref{GetStackTrace} for example usage.

\medskip
This function can be used in the \apieventref{SampleExecution} event to obtain the stack trace of the interrupted thread.
\end{apidesc}

\begin{apiphase}
\apiphaselive
\end{apiphase}

\begin{apicap}
\begin{itemize}
\item \apicapref{can_get_stack_trace_asynchronously} is necessary for this function to work.
\end{itemize}
\end{apicap}

\begin{apiparam}
\apiparamdef{depth}{Maximum depth of the call stack trace.}
\apiparamdef{max_frame_count}{Same as \code{depth}.}
\apiparamdef{frame_buffer}{On return, this buffer is filled with stack frame information.}
\apiparamdef{count_ptr}{On return, points to the number of records filled in.}
\end{apiparam}

\begin{apierror}
\apierrordef{JVMTIPROF_ERROR_INTERNAL}{It was not possible to obtain the stack trace.}
\apierrordef{JVMTIPROF_ERROR_ILLEGAL_ARGUMENT}{\code{depth} or \code{max_frame_count} is less than \code{0}.}
\apierrordef{JVMTIPROF_ERROR_ILLEGAL_ARGUMENT}{\code{depth} and \code{max_frame_count} differs.}
\apierrordef{JVMTIPROF_ERROR_NULL_POINTER}{\code{frame_buffer} or \code{count_ptr} is \code{NULL}.}
\end{apierror}
\end{apidef}
%%%%%%%%%%%%%%%%%%%%%%%%%%%%%%%%%%%%%%%%%%%%%%%%%%%%%%%%%%%%%%%%%%%%%%%%%%%%%%%

\section{Event Management} \label{sec:eventmgr}

%%%%%%%%%%%%%%%%%%%%%%%%%%%%%%%%%%%%%%%%%%%%%%%%%%%%%%%%%%%%%%%%%%%%%%%%%%%%%%%
\begin{apidef}{Set Event Notification Mode}
\begin{lstlisting}[language=C]
jvmtiProfError SetEventNotificationMode(
            jvmtiProfEnv* jvmtiprof_env,
            jvmtiEventMode mode,
            jvmtiProfEvent event_type,
            jthread event_thread,
            ...);
\end{lstlisting}

\begin{apidesc}
Controls the generation of events.

\medskip
If \code{JVMTI_ENABLE} is given in \code{mode}, the generation of events of type \code{event_type} are enabled. If \code{JVMTI_DISABLE} is given, the generation of events of \code{event_type} are disabled.

\medskip
If \code{event_thread} is \code{NULL}, the event is enabled or disabled globally; otherwise it is enabled or disabled for a particular thread. An event is generated for a particular thread if it is enabled either globally or at the thread level. No currently implemented event can be controlled at the thread level.

\medskip
See section~\ref{sec:eventmgr} for details on events.

\medskip
Initially no events are enabled at the thread level or global level.

\medskip
Any needed capabilities must be possessed before calling this function.
\end{apidesc}

\begin{apiphase}
\apiphaseonloadlive
\end{apiphase}

\begin{apicap}
Certain capabilities are necessary for some \code{event_type}s:
\begin{itemize}
\item \apicapref{can_generate_sample_execution_events} is necessary for \apieventref{SampleExecution} events.
\item \apicapref{can_intercept_methods} is necessary for \apieventref{SpecificMethodEntry} and \apieventref{SpecificMethodExit} events.
\end{itemize}
\end{apicap}

\begin{apiparam}
\apiparamdef{mode}{Whether to \code{JVMTI_ENABLE} or \code{JVMTI_DISABLE} the event.}
\apiparamdef{event_type}{The event to control.}
\apiparamdef{event_thread}{The thread to control. If \code{NULL}, the event is controlled globally.}
\apiparamdef{...}{For future expansion.}
\end{apiparam}

\begin{apierror}
\apierrordef{JVMTIPROF_ERROR_INVALID_THREAD}{\code{event_thread} is non-\code{NULL} and is not a valid thread}
\apierrordef{JVMTIPROF_ERROR_THREAD_NOT_ALIVE}{\code{event_thread} is non-\code{NULL} and is not live (has not been started or is now dead)}
\apierrordef{JVMTIPROF_ERROR_ILLEGAL_ARGUMENT}{Thread level control was attempted on events that do not allow thread control.}
\apierrordef{JVMTIPROF_ERROR_ILLEGAL_ARGUMENT}{Thread level control was attempted on events that do not allow thread control.}
\apierrordef{JVMTIPROF_ERROR_ILLEGAL_ARGUMENT}{\code{mode} is not a valid \code{jvmtiEventMode}.}
\apierrordef{JVMTIPROF_ERROR_ILLEGAL_ARGUMENT}{\code{event_type} is not a valid \code{jvmtiProfEvent}.}
\apierrordef{JVMTIPROF_ERROR_MUST_POSSESS_CAPABILITY}{The capability necessary to enable the events is not possessed.}
\end{apierror}
\end{apidef}
%%%%%%%%%%%%%%%%%%%%%%%%%%%%%%%%%%%%%%%%%%%%%%%%%%%%%%%%%%%%%%%%%%%%%%%%%%%%%%%
%%%%%%%%%%%%%%%%%%%%%%%%%%%%%%%%%%%%%%%%%%%%%%%%%%%%%%%%%%%%%%%%%%%%%%%%%%%%%%%
\begin{apidef}{Set Event Callbacks}
\begin{lstlisting}[language=C]
jvmtiProfError SetEventCallbacks(
            jvmtiProfEnv* jvmtiprof_env,
            const jvmtiProfEventCallbacks* callbacks,
            jint size_of_callbacks);
\end{lstlisting}

\begin{apidesc}
Sets the functions called for each event. The callbacks are specified by supplying a replacement function table. The function table is copied, changes to the local copy of the table will have no effect. This is an atomic action, all callbacks are set at once. No events are sent before this function is called. When an entry is \code{NULL} or when the event is beyond \code{size_of_callbacks} no event is sent.

\medskip
An event must be enabled and have a callback in order to be sent. The order in which this function and \apiref{SetEventNotificationMode} are called does not affect the result. 

\medskip
See section~\ref{sec:eventmgr} for details on events.
\end{apidesc}

\begin{apiphase}
\apiphaseonloadlive
\end{apiphase}

\begin{apicap}
\apicaprequired
\end{apicap}

\begin{apiparam}
\apiparamdef{callbacks}{The new event callbacks. If \code{NULL}, removes the existing callbacks.}
\apiparamdef{size_of_callbacks}{Must be equal \code{sizeof(jvmtiProfEventCallbacks)}. Used for version compatibility.}
\end{apiparam}

\begin{apierror}
\apierrordef{JVMTI_ERROR_ILLEGAL_ARGUMENT}{\code{size_of_callbacks} is less than \code{0} or not supported.}
\end{apierror}
\end{apidef}
%%%%%%%%%%%%%%%%%%%%%%%%%%%%%%%%%%%%%%%%%%%%%%%%%%%%%%%%%%%%%%%%%%%%%%%%%%%%%%%
%%%%%%%%%%%%%%%%%%%%%%%%%%%%%%%%%%%%%%%%%%%%%%%%%%%%%%%%%%%%%%%%%%%%%%%%%%%%%%%
\iffalse
\begin{apidef}{Generate Events}
\begin{lstlisting}[language=C]
jvmtiProfError GenerateEvents(
            jvmtiProfEnv* jvmtiprof_env,
            jvmtiProfEvent event_type);
\end{lstlisting}

\begin{apidesc}
\end{apidesc}

\begin{apiphase}
\end{apiphase}

\begin{apicap}
\end{apicap}

\begin{apiparam}
\apiparamdef{x}{Y}
\end{apiparam}

\begin{apierror}
\apierrordef{X}{Y}
\end{apierror}
\end{apidef}
\fi
%%%%%%%%%%%%%%%%%%%%%%%%%%%%%%%%%%%%%%%%%%%%%%%%%%%%%%%%%%%%%%%%%%%%%%%%%%%%%%%

\section{Capability}

%%%%%%%%%%%%%%%%%%%%%%%%%%%%%%%%%%%%%%%%%%%%%%%%%%%%%%%%%%%%%%%%%%%%%%%%%%%%%%%
\begin{apidef}{Get Potential Capabilities}
\begin{lstlisting}[language=C]
jvmtiProfError GetPotentialCapabilities(
            jvmtiProfEnv* jvmtiprof_env,
            jvmtiProfCapabilities* capabilities_ptr);
\end{lstlisting}

\begin{apidesc}
Returns the JVMTIPROF features that can potentially be possessed by this environment at this time. 

\medskip
The returned capabilities differ from the complete set of capabilities implemented by JVMTIPROF in two cases: another environment possesses capabilities that can only be possessed by one environment, or the current phase is live, and certain capabilities can only be added during the OnLoad phase. The \apiref{AddCapabilities} function may be used to set any or all of these capabilities. Currently possessed capabilities are included.

\medskip
Typically this function is used in the OnLoad function. Typically, a limited set of capabilities can be added in the live phase. In this case, the set of potentially available capabilities will likely differ from the OnLoad phase set. 

\end{apidesc}

\begin{apiphase}
\apiphaseonloadlive
\end{apiphase}

\begin{apicap}
\apicaprequired
\end{apicap}

\begin{apiparam}
\apiparamdef{capabilities_ptr}{Pointer through which the set of capabilities is returned.}
\end{apiparam}

\begin{apierror}
\apierrordef{JVMTIPROF_ERROR_NULL_POINTER}{\code{capabilities_ptr} is \code{NULL}}
\end{apierror}
\end{apidef}
%%%%%%%%%%%%%%%%%%%%%%%%%%%%%%%%%%%%%%%%%%%%%%%%%%%%%%%%%%%%%%%%%%%%%%%%%%%%%%%


%%%%%%%%%%%%%%%%%%%%%%%%%%%%%%%%%%%%%%%%%%%%%%%%%%%%%%%%%%%%%%%%%%%%%%%%%%%%%%%
\begin{apidef}{Add Capabilities}
\begin{lstlisting}[language=C]
jvmtiProfError AddCapabilities(
            jvmtiProfEnv* jvmtiprof_env,
            jvmtiProfCapabilities* capabilities_ptr);
\end{lstlisting}

\begin{apidesc}
Set new capabilities by adding the capabilities whose values are set to \code{1} in \code{*capabilities_ptr}. All previous capabilities are retained. Typically this function is used in the OnLoad function. A limited set of capabilities can be added in the live phase. 
\end{apidesc}

\begin{apiphase}
\apiphaseonloadlive
\end{apiphase}

\begin{apicap}
\apicaprequired
\end{apicap}

\begin{apiparam}
\apiparamdef{capabilities_ptr}{Pointer to the capabilities to be added.}
\end{apiparam}

\begin{apierror}
\apierrordef{JVMTIPROF_ERROR_NOT_AVAILABLE}{The desired capabilities are not even potentially available.}
\apierrordef{JVMTIPROF_ERROR_NULL_POINTER}{\code{capabilities_ptr} is \code{NULL}}
\end{apierror}
\end{apidef}
%%%%%%%%%%%%%%%%%%%%%%%%%%%%%%%%%%%%%%%%%%%%%%%%%%%%%%%%%%%%%%%%%%%%%%%%%%%%%%%
%%%%%%%%%%%%%%%%%%%%%%%%%%%%%%%%%%%%%%%%%%%%%%%%%%%%%%%%%%%%%%%%%%%%%%%%%%%%%%%
\begin{apidef}{Relinquish Capabilities}
\begin{lstlisting}[language=C]
jvmtiProfError RelinquishCapabilities(
            jvmtiProfEnv* jvmtiprof_env,
            const jvmtiProfCapabilities* capabilities_ptr);
\end{lstlisting}

\begin{apidesc}
Relinquish the capabilities whose values are set to \code{1} in \code{*capabilities_ptr}. Some implementations may allow only one environment to have a capability. This function releases capabilities so that they may be used by other environments. All other capabilities are retained. The capability will no longer be present in \apiref{GetCapabilities}. Attempting to relinquish a capability that the environment does not possess is not an error. 
\end{apidesc}

\begin{apiphase}
\apiphaseonloadlive
\end{apiphase}

\begin{apicap}
\apicaprequired
\end{apicap}

\begin{apiparam}
\apiparamdef{capabilities_ptr}{Pointer to the capabilities to relinquish.}
\end{apiparam}

\begin{apierror}
\apierrordef{JVMTIPROF_ERROR_NULL_POINTER}{\code{capabilities_ptr} is \code{NULL}}
\end{apierror}
\end{apidef}
%%%%%%%%%%%%%%%%%%%%%%%%%%%%%%%%%%%%%%%%%%%%%%%%%%%%%%%%%%%%%%%%%%%%%%%%%%%%%%%
%%%%%%%%%%%%%%%%%%%%%%%%%%%%%%%%%%%%%%%%%%%%%%%%%%%%%%%%%%%%%%%%%%%%%%%%%%%%%%%
\begin{apidef}{Get Capabilities}
\begin{lstlisting}[language=C]
jvmtiProfError GetCapabilities(
            jvmtiProfEnv* jvmtiprof_env,
            jvmtiProfCapabilities* capabilities_ptr);
\end{lstlisting}

\begin{apidesc}
Returns the optional JVMTIPROF features which this environment currently possesses. Each possessed capability is indicated by a \code{1} in the corresponding field of the capabilities structure. An environment does not possess a capability unless it has been successfully added with \apiref{AddCapabilities}. An environment only loses possession of a capability if it has been relinquished with \apiref{RelinquishCapabilities}. Thus, this function returns the net result of the \apiref{AddCapabilities} and \apiref{RelinquishCapabilities} calls which have been made. 
\end{apidesc}

\begin{apiphase}
\apiphaseany
\end{apiphase}

\begin{apicap}
\apicaprequired
\end{apicap}

\begin{apiparam}
\apiparamdef{capabilities_ptr}{Pointer through which the currently possessed capabilities are returned.}
\end{apiparam}

\begin{apierror}
\apierrordef{JVMTIPROF_ERROR_NULL_POINTER}{\code{capabilities_ptr} is \code{NULL}}
\end{apierror}
\end{apidef}
%%%%%%%%%%%%%%%%%%%%%%%%%%%%%%%%%%%%%%%%%%%%%%%%%%%%%%%%%%%%%%%%%%%%%%%%%%%%%%%

\section{General}

%%%%%%%%%%%%%%%%%%%%%%%%%%%%%%%%%%%%%%%%%%%%%%%%%%%%%%%%%%%%%%%%%%%%%%%%%%%%%%%
\begin{apidef}{Dispose Environment}
\begin{lstlisting}[language=C]
jvmtiProfError DisposeEnvironment(jvmtiProfEnv* jvmtiprof_env);
\end{lstlisting}

\begin{apidesc}
Shutdowns a JVMTIPROF connection created with \apiref{Create}. Disposes of any resources held by the environment. JVMTIPROF hooks on the associated JVMTI environment will be undone.

\medskip
Any capabilities held by this environment are relinquished.

\medskip
Events enabled by this environment will no longer be run. However, event handles currently running will continue to run. Caution must be exercised in the design of event handlers whose environment may be disposed of and thus become invalid during their execution.

\medskip
This environment may not be used after this call.
\end{apidesc}

\begin{apiphase}
\apiphaseany
\end{apiphase}

\begin{apicap}
\apicaprequired
\end{apicap}

\apiparamempty

\apierrorempty
\end{apidef}
%%%%%%%%%%%%%%%%%%%%%%%%%%%%%%%%%%%%%%%%%%%%%%%%%%%%%%%%%%%%%%%%%%%%%%%%%%%%%%%
%%%%%%%%%%%%%%%%%%%%%%%%%%%%%%%%%%%%%%%%%%%%%%%%%%%%%%%%%%%%%%%%%%%%%%%%%%%%%%%
\begin{apidef}{Get Jvmti Env}
\begin{lstlisting}[language=C]
jvmtiProfError GetJvmtiEnv(
            jvmtiProfEnv* jvmtiprof_env,
            jvmtiEnv** jvmti_env_ptr);
\end{lstlisting}

\begin{apidesc}
Obtains the JVMTI environment associated with this JVMTIPROF environment.
\end{apidesc}

\begin{apiphase}
\apiphaseany
\end{apiphase}

\begin{apicap}
\apicaprequired
\end{apicap}

\begin{apiparam}
\apiparamdef{jvmti_env_ptr}{Pointer through which the associated JVMTI environment is returned.}
\end{apiparam}

\begin{apierror}
\apierrordef{JVMTIPROF_ERROR_NULL_POINTER}{\code{jvmti_env_ptr} is \code{NULL}.}
\end{apierror}
\end{apidef}
%%%%%%%%%%%%%%%%%%%%%%%%%%%%%%%%%%%%%%%%%%%%%%%%%%%%%%%%%%%%%%%%%%%%%%%%%%%%%%%
%%%%%%%%%%%%%%%%%%%%%%%%%%%%%%%%%%%%%%%%%%%%%%%%%%%%%%%%%%%%%%%%%%%%%%%%%%%%%%%
\begin{apidef}{Get Version Number}
\begin{lstlisting}[language=C]
jvmtiProfError GetVersionNumber(
            jvmtiProfEnv* jvmtiprof_env,
            jint* version_ptr);
\end{lstlisting}

\begin{apidesc}
Return the JVMTIPROF version.

\medskip
The version identifier follows the same convention as JVMTI, with a major, minor and micro version. These can be accessed through the \code{JVMTI_VERSION_MASK_MAJOR}, \code{JVMTI_VERSION_MASK_MINOR}, and \code{JVMTI_VERSION_MASK_MICRO} bitmasks applied to the returned value. The version identifier does not include an interface type, as such \\ \code{JVMTI_VERSION_MASK_INTERFACE_TYPE} is not used.

\medskip
See JVMTI's \apijvmtiref{GetVersionNumber} for details on the bitmasks.
\end{apidesc}

\begin{apiphase}
\apiphaseany
\end{apiphase}

\begin{apicap}
\apicaprequired
\end{apicap}

\begin{apiparam}
\apiparamdef{version_ptr}{Pointer through which the JVMTIPROF version is returned.}
\end{apiparam}

\begin{apierror}
\apierrordef{JVMTIPROF_ERROR_NULL_POINTER}{\code{version_ptr} is \code{NULL}.}
\end{apierror}
\end{apidef}
%%%%%%%%%%%%%%%%%%%%%%%%%%%%%%%%%%%%%%%%%%%%%%%%%%%%%%%%%%%%%%%%%%%%%%%%%%%%%%%
%%%%%%%%%%%%%%%%%%%%%%%%%%%%%%%%%%%%%%%%%%%%%%%%%%%%%%%%%%%%%%%%%%%%%%%%%%%%%%%
\begin{apidef}{Get Error Name}
\begin{lstlisting}[language=C]
jvmtiProfError GetErrorName(
            jvmtiProfEnv* jvmtiprof_env,
            jvmtiProfError error,
            char** name_ptr);
\end{lstlisting}

\begin{apidesc}
Return the symbolic name for an \hyperref[api:ec]{error code}.

\medskip
For example \code{GetErrorName(env, JVMTIPROF_ERROR_NONE, &err_name)} would return in \code{err_name} the string \code{"JVMTIPROF_ERROR_NONE"}.
\end{apidesc}

\begin{apiphase}
\apiphaseany
\end{apiphase}

\begin{apicap}
\apicaprequired
\end{apicap}

\begin{apiparam}
\apiparamdef{error}{The error code.}
\apiparamdef{name_ptr}{Pointer through which the symbolic name is returned. The pointer must be freed with the associated JVMTI \apijvmtiref{Deallocate} function.}
\end{apiparam}

\begin{apierror}
\apierrordef{JVMTIPROF_ERROR_ILLEGAL_ARGUMENT}{\code{error} is not a valid error code.}
\apierrordef{JVMTIPROF_ERROR_NULL_POINTER}{\code{name_ptr} is \code{NULL}.}
\end{apierror}
\end{apidef}
%%%%%%%%%%%%%%%%%%%%%%%%%%%%%%%%%%%%%%%%%%%%%%%%%%%%%%%%%%%%%%%%%%%%%%%%%%%%%%%
%%%%%%%%%%%%%%%%%%%%%%%%%%%%%%%%%%%%%%%%%%%%%%%%%%%%%%%%%%%%%%%%%%%%%%%%%%%%%%%
\begin{apidef}{Set Verbose Flag}
\begin{lstlisting}[language=C]
jvmtiProfError SetVerboseFlag(
            jvmtiProfEnv* jvmtiprof_env,
            jvmtiProfVerboseFlag flag,
            jboolean value);
\end{lstlisting}

\begin{apidesc}
Control verbose output. This is the output which typically is sent to \code{stderr}.
\end{apidesc}

\begin{apiphase}
\apiphaseany
\end{apiphase}

\begin{apicap}
\apicaprequired
\end{apicap}

\begin{apiparam}
\apiparamdef{flag}{Which verbose flag to set.}
\apiparamdef{value}{New value of the flag.}
\end{apiparam}

\begin{apierror}
\apierrordef{JVMTIPROF_ERROR_ILLEGAL_ARGUMENT}{\code{flag} is not a valid flag. }
\end{apierror}
\end{apidef}
%%%%%%%%%%%%%%%%%%%%%%%%%%%%%%%%%%%%%%%%%%%%%%%%%%%%%%%%%%%%%%%%%%%%%%%%%%%%%%%

\section{Events}

\begin{apidef}{Specific Method Entry}
\begin{lstlisting}[language=C]
void JNICALL
SpecificMethodEntry(
    jvmtiProfEnv* jvmtiprof_env,
    jvmtiEnv* jvmti_env,
    JNIEnv* jni_env,
    jthread thread,
    jint hook_id)
\end{lstlisting}

\begin{apidesc}
This event is generated when Java programming language methods specified in \apiref{SetMethodEventFlag} are called.
\end{apidesc}

\begin{apiphase}
\eventphaselive
\end{apiphase}

\begin{eventtype}
\code{JVMTIPROF_EVENT_SPECIFIC_METHOD_ENTRY}
\end{eventtype}

\begin{apicap}
\apicapref{can_intercept_methods} is necessary for this event to be generated.
\end{apicap}

\begin{apiparam}
\apiparamdef{jni_env}{The JNI environment of the event (current) thread.}
\apiparamdef{thread}{Thread that is entering the method.}
\apiparamdef{hook_id}{Identifier of the method as returned by \apiref{SetMethodEventFlag}.}
\end{apiparam}
\end{apidef}

\begin{apidef}{Specific Method Exit}
\begin{lstlisting}[language=C]
void JNICALL
SpecificMethodExit(
    jvmtiProfEnv* jvmtiprof_env,
    jvmtiEnv* jvmti_env,
    JNIEnv* jni_env,
    jthread thread,
    jint hook_id)
\end{lstlisting}

\begin{apidesc}
This event is generated when Java programming language methods specified in \apiref{SetMethodEventFlag} return to the caller.
\end{apidesc}

\begin{apiphase}
\eventphaselive
\end{apiphase}

\begin{eventtype}
\code{JVMTIPROF_EVENT_SPECIFIC_METHOD_EXIT}
\end{eventtype}

\begin{apicap}
\apicapref{can_intercept_methods} is necessary for this event to be generated.
\end{apicap}

\begin{apiparam}
\apiparamdef{jni_env}{The JNI environment of the event (current) thread.}
\apiparamdef{thread}{Thread that is leaving the method.}
\apiparamdef{hook_id}{Identifier of the method as returned by \apiref{SetMethodEventFlag}.}
\end{apiparam}
\end{apidef}

\begin{apidef}{Sample Execution}
\begin{lstlisting}[language=C]
void JNICALL
SampleExecution(
    jvmtiProfEnv* jvmtiprof_env,
    jvmtiEnv* jvmti_env,
    JNIEnv* jni_env,
    jthread thread)
\end{lstlisting}

\begin{apidesc}
This event is generated when the CPU-time specified in \apiref{SetExecutionSamplingInterval} has elapsed. The thread that caused the timer to expire is interrupted, and this callback is invoked.

\medskip
The operations performed in this callback must be \href{https://man7.org/linux/man-pages/man7/signal-safety.7.html}{async-signal-safe}. It is only limited to primitive operations and a subset of system calls. This implies that the callback must not use standard C library functions or JVMTIPROF functions. It is recommended to publish any necessary information to an async-signal-safe queue and consume it in another thread.

\medskip
As an exception, the \apiref{GetStackTraceAsync} function can be used during this callback to obtain the stack trace of the interrupted thread.
\end{apidesc}

\begin{apiphase}
\eventphaselive
\end{apiphase}

\begin{eventtype}
\code{JVMTIPROF_EVENT_SAMPLE_EXECUTION}
\end{eventtype}

\begin{apicap}
\apicapref{can_generate_sample_execution_events} is necessary for this event to be generated.
\end{apicap}

\begin{apiparam}
\apiparamdef{jni_env}{The JNI environment of the event (current) thread.}
\apiparamdef{thread}{Thread that expired the timer.}
\end{apiparam}
\end{apidef}

\section{Error Codes} \label{api:ec}

The following error codes are equivalent to the ones \jvmtihref{ErrorSection}{presented in the JVMTI documentation}:
\begin{itemize}
\item{\code{JVMTIPROF_ERROR_NONE}}
\item{\code{JVMTIPROF_ERROR_NULL_POINTER}}
\item{\code{JVMTIPROF_ERROR_INVALID_ENVIRONMENT}}
\item{\code{JVMTIPROF_ERROR_WRONG_PHASE}}
\item{\code{JVMTIPROF_ERROR_INTERNAL}}
\item{\code{JVMTIPROF_ERROR_ILLEGAL_ARGUMENT}}
\item{\code{JVMTIPROF_ERROR_UNSUPPORTED_VERSION}}
\item{\code{JVMTIPROF_ERROR_NOT_AVAILABLE}}
\item{\code{JVMTIPROF_ERROR_MUST_POSSESS_CAPABILITY}}
\item{\code{JVMTIPROF_ERROR_UNATTACHED_THREAD}}
\end{itemize}



%% Parte pos-textual
\backmatter

% Bibliografia
% É aconselhável utilizar o BibTeX a partir de um arquivo, digamos "biblio.bib".
% Para ajuda na criação do arquivo .bib e utilização do BibTeX, recorra ao
% BibTeXpress em www.cin.ufpe.br/~paguso/bibtexpress
\bibliographystyle{abntex2-alf}
\bibliography{src/biblio}

% Apendices
% Comente se naoo houver apendices
%\iffalse
%\appendix

%\xchapter{Exemplo de Apêndice}{} %sem preambulo
%\lipsum
%\fi
% Eh aconselhavel criar cada apendice em um arquivo separado, digamos
% "apendice1.tex", "apendice.tex", ... "apendiceM.tex" e depois
% inclui--los com:
% \include{apendice1}
% \include{apendice2}
% ...
% \include{apendiceM}

%% Fim do documento
\end{document}
%------------------------------------------------------------------------------------------%
